\documentclass{article}
\usepackage[italian]{babel}
\usepackage{graphicx}
\usepackage{booktabs}
\usepackage{amsmath}
\usepackage{hyperref}
\usepackage[margin=1in]{geometry}

\title{Patologie del SNC e Controllo movimento}
\author{Orcam}
\date{}

\begin{document}

\maketitle

\section{Malattia di Alzheimer}
\subsection{Epidemiologia}
\begin{itemize}
    \item 50 milioni di casi globali (60\% in paesi a basso-medio reddito)
    \item Prevalenza: 5-8\% popolazione $\geq$60 anni
    \item Proiezione: 82 milioni (2030), 152 milioni (2050)
    \item Rappresenta 60-70\% dei casi di demenza
\end{itemize}

% Spazio per Figura 1: Distribuzione per età
\begin{figure}[h]
    \centering
    \includegraphics[width=0.7\textwidth]{figura1_distribuzione_eta}
    \caption{Distribuzione per età dei pazienti con Alzheimer (2019)}
    \label{fig:alzheimer_eta}
\end{figure}

\subsection{Sintomi}
10 segni principali:
\begin{enumerate}
    \item Perdita di memoria
    \item Difficoltà nelle attività quotidiane
    \item Problemi linguistici
    \item Disorientamento spazio-temporale
    \item Alterazione del giudizio
    \item Difficoltà nel ragionamento astratto
    \item Smarrimento oggetti
    \item Cambiamenti d'umore
    \item Problemi visuo-spaziali
    \item Ritiro sociale
\end{enumerate}

\subsection{Istopatologia}
Tre caratteristiche principali:
\begin{itemize}
    \item \textbf{Placche amiloidi}: Aggregati extracellulari di peptide Aβ
    \item \textbf{Grovigli neurofibrillari}: Aggregati intracellulari di proteina tau iperfosforilata
    \item \textbf{Atrofia cerebrale}: Perdita neuronale, ventricoli dilatati, solchi allargati
\end{itemize}

% Spazio per Figura 2: Confronto cervello sano/Alzheimer
\begin{figure}[h]
    \centering
    \includegraphics[width=0.7\textwidth]{figura2_atrofia_cerebrale}
    \caption{Confronto tra cervello sano e con Alzheimer avanzato}
    \label{fig:atrofia_cerebrale}
\end{figure}

\subsection{Fattori genetici}
\begin{itemize}
    \item \textbf{Sporadico (95\%)}: Gene di rischio ApoEε4
    \item \textbf{Familiare (5\%)}: Mutazioni in APP, PSEN1, PSEN2
\end{itemize}

\subsection{Terapie}
\begin{itemize}
    \item \textbf{Sintomatiche}: Inibitori colinesterasi (Donepezil), Antagonisti NMDA (Memantina)
    \item \textbf{Modificanti}: Anticorpi monoclonali anti-amiloide (Aducanumab, Lecanemab)
\end{itemize}

\section{Interfaccia Cervello-Macchina (BMI)}
\subsection{Definizione e funzioni}
Dispositivi che misurano/modificano l'attività neurale per:
\begin{itemize}
    \item Ripristinare capacità sensoriali/motorie
    \item Modulare attività patologica
    \item Controllare dispositivi esterni (arti robotici, computer)
\end{itemize}

\subsection{Implementazione}
\begin{itemize}
    \item Elettrodi impiantati in aree motorie
    \item Decodifica dell'attività neurale in comandi
    \item Applicazioni: controllo computer, arti robotici, stimolazione muscolare
\end{itemize}

% Spazio per Figura 3: Schema BMI
\begin{figure}[h]
    \centering
    \includegraphics[width=0.6\textwidth]{figura3_schema_bmi}
    \caption{Schema di interfaccia cervello-macchina per il controllo motorio}
    \label{fig:bmi_schema}
\end{figure}

\section{Sistema Motorio}
\subsection{Organizzazione gerarchica}
\begin{itemize}
    \item \textbf{Livello corticale}: Pianificazione movimento volontario
    \item \textbf{Gangli della base}: Selezione movimenti
    \item \textbf{Cervelletto}: Coordinazione e correzione errori
    \item \textbf{Tronco encefalico}: Controllo postura e movimenti automatici
    \item \textbf{Midollo spinale}: Circuiti locali per riflessi e CPG
\end{itemize}

% Spazio per Figura 4: Gerarchia motoria
\begin{figure}[h]
    \centering
    \includegraphics[width=0.8\textwidth]{figura4_gerarchia_motoria}
    \caption{Organizzazione gerarchica del sistema motorio}
    \label{fig:gerarchia_motoria}
\end{figure}

\subsection{Unità Motoria}
\begin{itemize}
    \item Motoneurone + fibre muscolari innervate
    \item \textbf{Principio della dimensione}: Reclutamento unità piccole $\rightarrow$ grandi
    \item Tipi fibre: Lente (S), Veloci resistenti (FR), Veloci affaticabili (FF)
\end{itemize}

\subsection{Riflessi}
\begin{itemize}
    \item \textbf{Monosinaptici}: Riflesso patellare (feedback antigravitazionale)
    \item \textbf{Polisinaptici}: Riflesso di ritrazione (risposta a stimoli nocicettivi)
\end{itemize}

\subsection{Movimenti Ritmici}
\begin{itemize}
    \item Controllati da Central Pattern Generators (CPG) spinali
    \item Attivati dalla Regione Locomotoria Mesencefalica (MLR)
    \item Esempi: Camminata, trotto, galoppo
\end{itemize}

% Spazio per Figura 5: CPG locomozione
\begin{figure}[h]
    \centering
    \includegraphics[width=0.7\textwidth]{figura5_cpg_locomozione}
    \caption{Generatori centrali di pattern per la locomozione}
    \label{fig:cpg_locomozione}
\end{figure}

\section{Patologie del Controllo Motorio}
\subsection{Sclerosi Laterale Amiotrofica (SLA)}
\begin{itemize}
    \item Degenerazione motoneuroni superiori e inferiori
    \item Sintomi: Debolezza muscolare, fascicolazioni, disfagia
    \item Terapie: Riluzolo, Edaravone, supporto ventilatorio
\end{itemize}

\subsection{Morbo di Parkinson}
\begin{itemize}
    \item Perdita neuroni dopaminergici nella substantia nigra
    \item Sintomi: Tremori, rigidità, bradicinesia
    \item Terapie: L-DOPA, stimolazione cerebrale profonda
\end{itemize}

\subsection{Atassie Cerebellari}
\begin{itemize}
    \item Deficit coordinazione motoria e equilibrio
    \item Cause: Ictus, traumi, mutazioni genetiche (es. SCA)
\end{itemize}

% Spazio per Figura 6: Confronto sindromi motorie
\begin{figure}[h]
    \centering
    \includegraphics[width=0.9\textwidth]{figura6_sindromi_motorie}
    \caption{Confronto tra lesioni motoneuroni superiori e inferiori}
    \label{fig:sindromi_motorie}
\end{figure}

\end{document}