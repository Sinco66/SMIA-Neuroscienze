\documentclass[12pt]{article}
\usepackage[italian]{babel}
\usepackage{graphicx}
\usepackage{amsmath}
\usepackage{amssymb}
\usepackage{booktabs}
\usepackage[margin=2cm]{geometry}
\usepackage{subcaption}

\title{Plasticità Sinaptica}
\author{Orcam}
\date{}

\begin{document}

\maketitle

\section{Introduzione alla Plasticità}
\begin{itemize}
    \item Capacità del cervello di modificare struttura e funzione in risposta all'esperienza
    \item Tipi di plasticità:
    \begin{itemize}
        \item Funzionale: cambiamenti nell'efficacia sinaptica
        \item Strutturale: modifiche morfologiche
    \end{itemize}
    \item Persiste in età adulta e nell'invecchiamento
\end{itemize}

\section{Plasticità a Breve Termine}
\subsection{Meccanismi Presinaptici}
\begin{itemize}
    \item \textbf{Facilitazione sinaptica}: aumento EPSP per accumulo Ca$^{2+}$
    \item \textbf{Depressione sinaptica}: riduzione EPSP per esaurimento vescicole
    \item \textbf{Potenziamento post-tetanico (PTP)}: aumento duraturo dopo stimolazione ad alta frequenza
    \item Mediata da cambiamenti nella probabilità di rilascio
\end{itemize}

% Spazio per immagine: Facilitazione sinaptica (Pag. 8)
\begin{figure}[h]
    \centering
    \includegraphics[width=0.7\textwidth]{facilitazione_sinaptica}
    \caption{Meccanismo della facilitazione sinaptica}
\end{figure}

\section{Modulazione Presinaptica}
\begin{itemize}
    \item \textbf{Facilitazione presinaptica}: 
    \begin{itemize}
        \item Chiusura canali K$^+$ $\rightarrow$ prolungamento PA
        \item Maggior ingresso Ca$^{2+}$ $\rightarrow$ aumento rilascio NT
    \end{itemize}
    \item \textbf{Inibizione presinaptica}: 
    \begin{itemize}
        \item Apertura canali K$^+$/Cl$^-$ $\rightarrow$ riduzione ingresso Ca$^{2+}$
        \item Diminuzione rilascio NT
    \end{itemize}
    \item Differenze con inibizione postsinaptica
\end{itemize}

% Spazio per immagine: Modulazione presinaptica (Pag. 18)
\begin{figure}[h]
    \centering
    \includegraphics[width=0.8\textwidth]{modulazione_presinaptica}
    \caption{Confronto tra facilitazione e inibizione presinaptica}
\end{figure}

\section{Plasticità a Lungo Termine}
\subsection{Potenziamento a Lungo Termine (LTP)}
\begin{itemize}
    \item Scoperto nell'ippocampo (Bliss e Lomo)
    \item Indotto da stimolazione ad alta frequenza
    \item Proprietà:
    \begin{itemize}
        \item Specificità: solo sinapsi attivate
        \item Associatività: sinapsi deboli + forti
        \item Coincidenza: attivazione pre+post
    \end{itemize}
    \item Meccanismo: recettori NMDA come rilevatori di coincidenza
\end{itemize}

% Spazio per immagine: LTP nell'ippocampo (Pag. 24)
\begin{figure}[h]
    \centering
    \includegraphics[width=0.9\textwidth]{ltp_ippocampo}
    \caption{Meccanismo del LTP nelle sinapsi CA3-CA1}
\end{figure}

\subsection{Depressione a Lungo Termine (LTD)}
\begin{itemize}
    \item Indotto da stimolazione a bassa frequenza
    \item Meccanismo: internalizzazione recettori AMPA
    \item Dipendenza dal Ca$^{2+}$: bassi livelli attivano fosfatasi
\end{itemize}

\section{Basi Molecolari}
\subsection{Ruolo del Calcio}
\begin{itemize}
    \item Alta [Ca$^{2+}$] $\rightarrow$ attivazione chinasi (LTP)
    \item Bassa [Ca$^{2+}$] $\rightarrow$ attivazione fosfatasi (LTD)
\end{itemize}

\subsection{Modifiche Geniche}
\begin{itemize}
    \item Attivazione di CREB e fattori di trascrizione
    \item Sintesi di proteine per:
    \begin{itemize}
        \item Crescita sinaptica
        \item Formazione nuove spine dendritiche
    \end{itemize}
\end{itemize}

% Spazio per immagine: Cascata CREB (Pag. 34)
\begin{figure}[h]
    \centering
    \includegraphics[width=0.8\textwidth]{cascata_creb}
    \caption{Meccanismi di trascrizione attività-dipendente}
\end{figure}

\section{Plasticità Strutturale}
\begin{itemize}
    \item Cambiamenti morfologici:
    \begin{itemize}
        \item Spine dendritiche (numero, forma, dimensione)
        \item Arborizzazione dendritica
    \end{itemize}
    \item Tipi di spine: filopodi, sottili, a fungo, tozze
    \item Alterazioni in patologie (Alzheimer, schizofrenia)
\end{itemize}

% Spazio per immagine: Spine dendritiche (Pag. 44)
\begin{figure}[h]
    \centering
    \includegraphics[width=0.7\textwidth]{spine_dendritiche}
    \caption{Classificazione morfologica delle spine dendritiche}
\end{figure}

\section{Plasticità nei Circuiti Specifici}
\subsection{Corteccia Cerebellare}
\begin{itemize}
    \item Meccanismo LTD: 
    \begin{itemize}
        \item Co-attivazione fibre parallele e rampicanti
        \item Internalizzazione recettori AMPA
    \end{itemize}
    \item Ruolo nell'apprendimento motorio
\end{itemize}

\end{document}