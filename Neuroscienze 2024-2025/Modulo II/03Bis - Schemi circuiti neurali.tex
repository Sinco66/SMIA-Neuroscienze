\documentclass[12pt]{article}
\usepackage[italian]{babel}
\usepackage{graphicx}
\usepackage{amsmath}
\usepackage{amssymb}
\usepackage{booktabs}
\usepackage[margin=2cm]{geometry}
\usepackage{subcaption}

\title{Schemi Circuiti Neurali}
\author{Orcam}
\date{}

\begin{document}

\maketitle

\section{Elementi Fondamentali dei Circuiti}
\begin{itemize}
    \item \textbf{Neuroni afferenti}: Input (periferia $\rightarrow$ SNC)
    \item \textbf{Neuroni efferenti}: Output (SNC $\rightarrow$ periferia)
    \item \textbf{Neuroni di proiezione}: Collegano regioni diverse
    \item \textbf{Interneuroni}: Circuiti locali (intra-regionali)
\end{itemize}

\section{Principi di Connessione}
\subsection{Convergenza e Divergenza}
\begin{itemize}
    \item \textbf{Convergenza}: Molti input $\rightarrow$ Pochi output
    \item \textbf{Divergenza}: Pochi input $\rightarrow$ Molti output
    \item Esempio sistema visivo: 100M recettori $\rightarrow$ 1M gangli $\rightarrow$ miliardi corticali
\end{itemize}

% Spazio per immagine: Convergenza/divergenza (Pag. 4)
\begin{figure}[h]
    \centering
    \includegraphics[width=0.9\textwidth]{sistema_visivo}
    \caption{Convergenza e divergenza nel sistema visivo}
\end{figure}

\section{Circuiti Eccitatori}
\subsection{Eccitazione Feedforward}
\begin{itemize}
    \item Propagazione sequenziale: A $\rightarrow$ B $\rightarrow$ C
    \item Meccanismo: Propagazione lineare
    \item Funzione: Trasmissione informazioni
\end{itemize}

\subsection{Eccitazione Feedback}
\begin{itemize}
    \item Retroazione: Output $\rightarrow$ Rinforzo input
    \item Meccanismo: Ripetizione
    \item Funzione: Rinforzo segnale
\end{itemize}

\subsection{Eccitazione Ricorrente}
\begin{itemize}
    \item Auto-rinforzo: Neuroni si eccitano reciprocamente
    \item Meccanismo: Amplificazione
    \item Funzione: Stabilità rappresentazione
\end{itemize}

\subsection{Eccitazione Ricorrente Laterale}
\begin{itemize}
    \item Collegamento circuiti paralleli
    \item Propagazione orizzontale (stesso livello)
    \item Sincronizzazione attività
\end{itemize}

% Spazio per immagine: Schemi eccitatori (Pag. 5)
\begin{figure}[h]
    \centering
    \includegraphics[width=0.8\textwidth]{circuiti_eccitatori}
    \caption{Schemi fondamentali di circuiti eccitatori}
\end{figure}

\section{Circuiti Inibitori}
\subsection{Inibizione Feedforward}
\begin{itemize}
    \item Meccanismo: Eccitazione $\rightarrow$ Interneurone inibitorio $\rightarrow$ Inibizione
    \item Funzione: Arresto/limitazione eccitazione
\end{itemize}

\subsection{Inibizione Feedback}
\begin{itemize}
    \item Meccanismo: Output $\rightarrow$ Interneurone inibitorio $\rightarrow$ Autoregolazione
    \item Funzione: Normalizzazione output
\end{itemize}

\subsection{Disinibizione}
\begin{itemize}
    \item Meccanismo: Inibitore 1 $\rightarrow$ Inibitore 2 $\rightarrow$ Attivazione bersaglio
    \item Funzione: Controllo a doppia negazione
\end{itemize}

\subsection{Inibizione Ricorrente/Laterale}
\begin{itemize}
    \item Inibizione trasversale: Circuiti paralleli si inibiscono
    \item Amplificazione differenze
    \item Creazione oscillazioni sincrone
\end{itemize}

% Spazio per immagine: Schemi inibitori (Pag. 10)
\begin{figure}[h]
    \centering
    \includegraphics[width=0.8\textwidth]{circuiti_inibitori}
    \caption{Schemi fondamentali di circuiti inibitori}
\end{figure}

\section{Esempi Clinici}
\subsection{Riflesso Patellare}
\begin{itemize}
    \item Arco riflesso monosinaptico
    \item Componenti:
    \begin{itemize}
        \item Recettore sensoriale (muscolo estensore)
        \item Neurone afferente
        \item Motoneurone estensore (eccitazione diretta)
        \item Interneurone inibitorio (muscolo flessore)
    \end{itemize}
    \item Funzione: Estensione gamba + rilassamento flessori
\end{itemize}

% Spazio per immagine: Riflesso patellare (Pag. 13-15)
\begin{figure}[h]
    \centering
    \includegraphics[width=0.7\textwidth]{riflesso_patellare}
    \caption{Circuito del riflesso patellare con inibizione feedforward}
\end{figure}

\subsection{Circuito dei Gangli della Base}
\begin{itemize}
    \item Esempio di disinibizione
    \item Meccanismo: Corteccia $\rightarrow$ Striato $\rightarrow$ Globo pallido $\rightarrow$ Talamo
    \item Funzione: Controllo movimento
\end{itemize}

% Spazio per immagine: Disinibizione (Pag. 17)
\begin{figure}[h]
    \centering
    \includegraphics[width=0.8\textwidth]{circuito_disinibizione}
    \caption{Circuito di disinibizione nei gangli della base}
\end{figure}

\end{document}