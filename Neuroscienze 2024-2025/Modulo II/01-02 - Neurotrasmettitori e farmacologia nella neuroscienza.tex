\documentclass[12pt]{article}
\usepackage[italian]{babel}
\usepackage{graphicx}
\usepackage{amsmath}
\usepackage{amssymb}
\usepackage{booktabs}
\usepackage[margin=2cm]{geometry}
\usepackage{subcaption}

\title{Neurotrasmettitori: Fondamenti di Neuroscienze}
\author{Orcam}
\date{}

\begin{document}

\maketitle

\section{Introduzione alle Reti Neurali}
\begin{itemize}
    \item 100 miliardi di neuroni con $\sim$1000 trilioni di sinapsi
    \item Neuroscienze: disciplina multidisciplinare che integra:
    \begin{itemize}
        \item Neuroanatomia, Neurofisiologia, Neurochimica
        \item Psicologia, Psichiatria, Neurologia
    \end{itemize}
    \item Interazione con l'IA:
    \begin{itemize}
        \item Neuroscienze $\rightarrow$ Ispirazione algoritmi IA
        \item IA $\rightarrow$ Analisi dati neuroscientifici
    \end{itemize}
\end{itemize}

% Spazio per immagine: Reti neurali biologiche vs artificiali (Pag. 5)
\begin{figure}[h]
    \centering
    \includegraphics[width=0.8\textwidth]{reti_neurali_comparazione}
    \caption{Confronto tra reti neurali biologiche e artificiali}
\end{figure}

\section{Fondamenti di Neurotrasmissione}
\subsection{Definizione di Neurotrasmettitore}
Sostanza che:
\begin{enumerate}
    \item È presente nel neurone presinaptico
    \item Viene rilasciata in modo Ca$^{2+}$-dipendente
    \item Lega recettori specifici postsinaptici
\end{enumerate}

\subsection{Meccanismo di Rilascio}
\begin{enumerate}
    \item Potenziale d'azione invade terminazione presinaptica
    \item Apertura canali Ca$^{2+}$ voltaggio-dipendenti
    \item Fusione vescicole con membrana presinaptica
    \item Rilascio neurotrasmettitore per esocitosi
\end{enumerate}

% Spazio per immagine: Processo sinaptico (Pag. 11)
\begin{figure}[h]
    \centering
    \includegraphics[width=0.7\textwidth]{meccanismo_sinaptico}
    \caption{Meccanismo di trasmissione sinaptica}
\end{figure}

\subsection{Recettori}
\begin{table}[h]
    \centering
    \begin{tabular}{lll}
        \toprule
        \textbf{Tipo} & \textbf{Meccanismo} & \textbf{Esempi} \\
        \midrule
        Ionotropico & Canali ionici rapidi & nAChR, GABA$_A$ \\
        Metabotropico & Secondi messaggeri (lenti) & mAChR, rec. dopamina \\
        \bottomrule
    \end{tabular}
    \caption{Classificazione recettori}
\end{table}

\section{Principali Neurotrasmettitori}
\subsection{Acetilcolina (ACh)}
\begin{itemize}
    \item Primo NT identificato
    \item Recettori:
    \begin{itemize}
        \item Nicotinici (nAChR): ionotropici
        \item Muscarinici (mAChR): metabotropici
    \end{itemize}
    \item Ruolo: controllo muscolare, memoria
    \item Patologia: Miastenia Gravis (autoimmune)
\end{itemize}

% Spazio per immagine: Recettori nicotinici (Pag. 25)
\begin{figure}[h]
    \centering
    \includegraphics[width=0.6\textwidth]{recettori_acetilcolina}
    \caption{Struttura recettori nicotinici}
\end{figure}

\subsection{Glutammato}
\begin{itemize}
    \item NT eccitatorio principale
    \item Recettori ionotropici:
    \begin{itemize}
        \item AMPA, NMDA, Kainato
        \item NMDA: permeabile a Ca$^{2+}$, bloccato da Mg$^{2+}$
    \end{itemize}
    \item Ruolo: apprendimento, plasticità sinaptica
\end{itemize}

\subsection{GABA e Glicina}
\begin{itemize}
    \item NT inibitori principali
    \item GABA$_A$: canale per Cl$^-$ (ionotropico)
    \item Durante sviluppo: effetto eccitatorio
    \item Farmaci: benzodiazepine, barbiturici
\end{itemize}

\subsection{Ammine Biogene}
\begin{table}[h]
    \centering
    \begin{tabular}{lll}
        \toprule
        \textbf{NT} & \textbf{Recettori} & \textbf{Funzione} \\
        \midrule
        Dopamina & D1-like (G$_s$), D2-like (G$_i$) & Movimento, ricompensa \\
        Serotonina & 7 famiglie (5-HT$_{1-7}$) & Umore, sonno \\
        Noradrenalina & $\alpha$, $\beta$ adrenergici & Attenzione, stress \\
        Istamina & H$_{1-4}$ & Veglia, allergie \\
        \bottomrule
    \end{tabular}
    \caption{Ammine biogene e loro funzioni}
\end{table}

% Spazio per immagine: Metabolismo dopamina (Pag. 48)
\begin{figure}[h]
    \centering
    \includegraphics[width=0.7\textwidth]{metabolismo_dopamina}
    \caption{Via metabolica della dopamina}
\end{figure}

\section{Altri Neurotrasmettitori}
\subsection{Peptidi}
\begin{itemize}
    \item Esempi: encefaline, sostanza P, ossitocina
    \item Rilascio lento e diffuso
    \item Funzione: neuromodulazione, dolore, emozioni
    \item Degradati da peptidasi
\end{itemize}

\subsection{Non Convenzionali}
\begin{itemize}
    \item Endocannabinoidi:
    \begin{itemize}
        \item Retrograde signaling
        \item Recettori CB1 (SNC), CB2 (periferico)
    \end{itemize}
    \item NO (monossido di azoto):
    \begin{itemize}
        \item Diffusione trans-membrana
        \item Attiva guanilato ciclasi
    \end{itemize}
\end{itemize}

\section{Farmacologia}
\subsection{Meccanismi d'Azione}
\begin{itemize}
    \item Agonisti: mimano l'NT (es. nicotina per nAChR)
    \item Antagonisti: bloccano l'NT (es. curaro per nAChR)
    \item Modificano: sintesi, rilascio, ricaptazione, degradazione
\end{itemize}

\subsection{Esempi Clinici}
\begin{itemize}
    \item Miastenia Gravis: anticorpi anti-nAChR
    \item Cocaina: inibisce DAT (dopamina)
    \item SSRIs: inibiscono ricaptazione serotonina
    \item Benzodiazepine: potenziano GABA$_A$
\end{itemize}

% Spazio per immagine: Agonisti/antagonisti (Pag. 31)
\begin{figure}[h]
    \centering
    \includegraphics[width=0.8\textwidth]{farmaci_meccanismo}
    \caption{Meccanismi d'azione di farmaci}
\end{figure}

\section{Connessioni con l'Intelligenza Artificiale}
\begin{itemize}
    \item Analogie:
    \begin{itemize}
        \item Sinapsi biologiche $\leftrightarrow$ Pesi nelle reti neurali
        \item Eccitazione/inibizione $\leftrightarrow$ Pesi positivi/negativi
    \end{itemize}
    \item Applicazioni IA in neuroscienze:
    \begin{itemize}
        \item Drug discovery accelerato
        \item Simulazioni binding molecolare
        \item Medicina personalizzata
    \end{itemize}
    \item Strumenti educativi: Neuronify (simulazione reti)
\end{itemize}

\end{document}