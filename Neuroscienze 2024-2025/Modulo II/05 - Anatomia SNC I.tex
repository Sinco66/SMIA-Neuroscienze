\documentclass[12pt]{article}
\usepackage[italian]{babel}
\usepackage{graphicx}
\usepackage{amsmath}
\usepackage{amssymb}
\usepackage{booktabs}
\usepackage[margin=2cm]{geometry}
\usepackage{subcaption}
\usepackage{multirow}

\title{Anatomia SNC I}
\author{Orcam}
\date{}

\begin{document}

\maketitle

\section{Organizzazione del SNC}
\begin{itemize}
    \item Sistema Nervoso Centrale (SNC): cervello + midollo spinale
    \item Sistema Nervoso Periferico (SNP): somatico + autonomo
    \item Riferimenti anatomici:
    \begin{itemize}
        \item Assi: rostro-caudale, dorso-ventrale, medio-laterale
        \item Piani: sagittale, coronale, orizzontale
    \end{itemize}
\end{itemize}

% Spazio per immagine: Riferimenti anatomici (Pag. 6-9)
\begin{figure}[h]
    \centering
    \includegraphics[width=0.8\textwidth]{assi_anatomici}
    \caption{Sistema di riferimento neuroanatomico}
\end{figure}

\section{Componenti Cellulari}
\begin{itemize}
    \item \textbf{Neuroni}: unità funzionali
    \item \textbf{Glia}: astrociti, oligodendrociti, microglia
    \item \textbf{Sostanza grigia}: corpi cellulari
    \item \textbf{Sostanza bianca}: assoni mielinizzati
\end{itemize}

\section{Organizzazione Macroscopica}
\begin{table}[h]
    \centering
    \begin{tabular}{lp{10cm}}
        \toprule
        \textbf{Struttura} & \textbf{Descrizione} \\
        \midrule
        Nucleo & Gruppo di neuroni con funzioni simili \\
        Tratto & Fascio di assoni che connette regioni \\
        Commissura & Fibre che connettono emisferi \\
        \bottomrule
    \end{tabular}
    \caption{Termini neuroanatomici fondamentali}
\end{table}

\section{Encefalo}
\subsection{Divisioni Principali}
\begin{enumerate}
    \item Emisferi cerebrali
    \item Tronco encefalico (mesencefalo, ponte, bulbo)
    \item Cervelletto
    \item Diencefalo (talamo, ipotalamo)
\end{enumerate}

% Spazio per immagine: Divisioni encefalo (Pag. 22)
\begin{figure}[h]
    \centering
    \includegraphics[width=0.7\textwidth]{divisioni_encefalo}
    \caption{Origine embrionale delle strutture cerebrali}
\end{figure}

\section{Corteccia Cerebrale}
\subsection{Lobi e Solchi}
\begin{itemize}
    \item Lobi: frontale, parietale, temporale, occipitale
    \item Solchi principali: centrale, laterale, parieto-occipitale
    \item Giri: precentrale, postcentrale, frontali, temporali
\end{itemize}

% Spazio per immagine: Lobi cerebrali (Pag. 24)
\begin{figure}[h]
    \centering
    \includegraphics[width=0.9\textwidth]{lobi_cerebrali}
    \caption{Suddivisione in lobi della corteccia cerebrale}
\end{figure}

\subsection{Organizzazione Istologica}
\begin{itemize}
    \item Neocorteccia (6 strati)
    \item Paleocorteccia (3-5 strati)
    \item Archicorteccia (3 strati)
    \item Organizzazione colonnare
\end{itemize}

\section{Strutture Sottocorticali}
\subsection{Gangli della Base}
\begin{itemize}
    \item Componenti: nucleo caudato, putamen, globo pallido
    \item Funzioni: controllo motorio, apprendimento, motivazione
    \item Circuiti: via diretta/inversa
\end{itemize}

% Spazio per immagine: Gangli della base (Pag. 36)
\begin{figure}[h]
    \centering
    \includegraphics[width=0.7\textwidth]{gangli_base}
    \caption{Componenti dei nuclei della base}
\end{figure}

\subsection{Sistema Limbico}
\begin{itemize}
    \item Componenti: ippocampo, amigdala, giro cingolato
    \item Funzioni: emozioni, memoria, apprendimento
\end{itemize}

% Spazio per immagine: Sistema limbico (Pag. 39)
\begin{figure}[h]
    \centering
    \includegraphics[width=0.8\textwidth]{sistema_limbico}
    \caption{Strutture del sistema limbico}
\end{figure}

\subsection{Diencefalo}
\begin{itemize}
    \item \textbf{Talamo}: stazione di relay sensoriale
    \item \textbf{Ipotalamo}: regolazione omeostatica
    \item Epitalamo (ghiandola pineale)
    \item Subtalamo
\end{itemize}

\section{Tronco Encefalico}
\begin{itemize}
    \item \textbf{Mesencefalo}: collicoli, sostanza nera, nucleo rosso
    \item \textbf{Ponte}: nuclei dei nervi cranici
    \item \textbf{Bulbo}: controllo funzioni vitali
    \item Funzioni: 
    \begin{itemize}
        \item Via di passaggio per fasci ascendenti/discendenti
        \item Origine nervi cranici
        \item Formazione reticolare (veglia)
    \end{itemize}
\end{itemize}

% Spazio per immagine: Tronco encefalico (Pag. 45)
\begin{figure}[h]
    \centering
    \includegraphics[width=0.8\textwidth]{tronco_encefalico}
    \caption{Componenti del tronco encefalico}
\end{figure}

\section{Cervelletto}
\begin{itemize}
    \item Struttura: corteccia cerebellare + nuclei profondi
    \item Divisioni: vestibolare, spinocerebellare, cerebrocerebellare
    \item Funzioni: coordinazione motoria, apprendimento motorio
\end{itemize}

\section{Midollo Spinale}
\begin{itemize}
    \item Organizzazione:
    \begin{itemize}
        \item Sostanza grigia: corna dorsali (sensitive), ventrali (motorie)
        \item Sostanza bianca: colonne ascendenti/discendenti
    \end{itemize}
    \item Rigonfiamenti cervicale e lombare
    \item 31 paia di nervi spinali
\end{itemize}

% Spazio per immagine: Midollo spinale (Pag. 55)
\begin{figure}[h]
    \centering
    \includegraphics[width=0.7\textwidth]{midollo_spinale}
    \caption{Organizzazione del midollo spinale}
\end{figure}

\section{Risorse Digitali}
\begin{itemize}
    \item Allen Brain Atlas: dati di espressione genica
    \item NeuroMorpho.Org: ricostruzioni neuronali 3D
    \item BrainFacts.org: risorse educative
\end{itemize}

\end{document}