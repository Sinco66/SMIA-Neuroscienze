\documentclass[12pt]{article}
\usepackage[italian]{babel}
\usepackage{graphicx}
\usepackage{amsmath}
\usepackage{amssymb}
\usepackage{booktabs}
\usepackage[margin=2cm]{geometry}
\usepackage{subcaption}

\title{Circuiti Neurali}
\author{Orcam}
\date{}

\begin{document}

\maketitle

\section{Introduzione ai Circuiti Neurali}
\begin{itemize}
    \item 100 miliardi di neuroni nel SNC umano
    \item $\sim$1000 connessioni/neurone $\rightarrow$ 100.000 miliardi di sinapsi
    \item Il neurone: unità di elaborazione delle informazioni
    \item Computazione neurale: trasformazione input multipli in output singolo
\end{itemize}

% Spazio per immagine: Schema cervello e neurone (Pag. 3)
\begin{figure}[h]
    \centering
    \includegraphics[width=0.8\textwidth]{neurone_circuito}
    \caption{Dal neurone al circuito neurale}
\end{figure}

\section{Integrazione Neurale}
\subsection{Meccanismi di Sommazione}
\begin{itemize}
    \item \textbf{Sommazione spaziale}: combinazione di EPSP da sinapsi diverse
    \item \textbf{Sommazione temporale}: combinazione di EPSP dalla stessa sinapsi in rapida successione
    \item L'ampiezza della depolarizzazione determina la frequenza dei potenziali d'azione
\end{itemize}

% Spazio per immagine: Tipi di sinapsi (Pag. 8)
\begin{figure}[h]
    \centering
    \includegraphics[width=0.7\textwidth]{tipi_sinapsi}
    \caption{Sinapsi asimmetriche (eccitatorie) e simmetriche (inibitorie)}
\end{figure}

\section{Diversità Neuronale}
\subsection{Classi Morfologiche}
\begin{itemize}
    \item Bipolari, pseudounipolari, multipolari, unipolari
    \item Cellule piramidali, stellate, a canestro, fusiformi
\end{itemize}

\subsection{Banche Dati Neuronali}
\begin{itemize}
    \item \textbf{NeuroMorpho.Org}: 147.000+ ricostruzioni digitali
    \item \textbf{Allen Brain Atlas}: dati elettrofisiologici, morfologici e trascrittomici
    \item \textbf{ModelDB}: modelli computazionali di neuroni
\end{itemize}

% Spazio per immagine: Diversità neuronale (Pag. 16)
\begin{figure}[h]
    \centering
    \includegraphics[width=0.9\textwidth]{diversita_neuronale}
    \caption{Esempi di ricostruzioni digitali da NeuroMorpho.Org}
\end{figure}

\section{Elementi dei Circuiti Neurali}
\begin{table}[h]
    \centering
    \begin{tabular}{lp{8cm}}
        \toprule
        \textbf{Elemento} & \textbf{Funzione} \\
        \midrule
        Neuroni afferenti & Portano informazioni \textit{verso} una regione (input) \\
        Neuroni efferenti & Portano informazioni \textit{da} una regione (output) \\
        Neuroni di proiezione & Collegano regioni diverse (inter-regionali) \\
        Interneuroni & Circuiti locali (intra-regionali) \\
        \bottomrule
    \end{tabular}
    \caption{Componenti fondamentali dei circuiti neurali}
\end{table}

\section{Principi di Connessione}
\subsection{Convergenza e Divergenza}
\begin{itemize}
    \item \textbf{Convergenza}: Molti neuroni presinaptici $\rightarrow$ Pochi postsinaptici
    \item \textbf{Divergenza}: Pochi neuroni presinaptici $\rightarrow$ Molti postsinaptici
    \item Esempio: Sistema visivo (100M recettori $\rightarrow$ 1M gangli $\rightarrow$ miliardi corticali)
\end{itemize}

% Spazio per immagine: Convergenza/divergenza (Pag. 25)
\begin{figure}[h]
    \centering
    \includegraphics[width=0.8\textwidth]{convergenza_divergenza}
    \caption{Convergenza e divergenza nel sistema visivo}
\end{figure}

\section{Motivi Circuitali Fondamentali}
\subsection{Circuiti Eccitatori}
\begin{itemize}
    \item Convergenza
    \item Divergenza
    \item Feedforward
    \item Feedback
    \item Ricorrenza laterale
\end{itemize}

\subsection{Circuiti Inibitori}
\begin{itemize}
    \item Inibizione feedforward
    \item Inibizione feedback
    \item Inibizione ricorrente
    \item Inibizione laterale
    \item Disinibizione
\end{itemize}

% Spazio per immagine: Schemi circuiti (Pag. 26-27)
\begin{figure}[h]
    \centering
    \begin{subfigure}{0.45\textwidth}
        \centering
        \includegraphics[width=\textwidth]{circuiti_eccitatori}
        \caption{Circuiti eccitatori}
    \end{subfigure}
    \hfill
    \begin{subfigure}{0.45\textwidth}
        \centering
        \includegraphics[width=\textwidth]{circuiti_inibitori}
        \caption{Circuiti inibitori}
    \end{subfigure}
    \caption{Schemi fondamentali di circuiti neurali}
\end{figure}

\section{Strumenti Computazionali}
\begin{itemize}
    \item \textbf{Neuronify}: Simulatore educativo di reti neurali
    \item \textbf{Allen SDK}: Accesso programmatico a dati neuronali
    \item \textbf{ModelDB}: Repository di modelli computazionali
\end{itemize}

\end{document}