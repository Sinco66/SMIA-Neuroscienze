\documentclass[11pt]{article}
\usepackage[italian]{babel}
\usepackage[utf8]{inputenc}
\usepackage[T1]{fontenc}
\usepackage{geometry}
\usepackage{graphicx}
\usepackage{amsmath}
\usepackage{amssymb}
\usepackage{booktabs}
\usepackage{multirow}
\usepackage{enumitem}
\geometry{a4paper, left=2cm, right=2cm, top=2cm, bottom=2.5cm}

\title{Memoria II}
\author{Orcam}
\date{}

\begin{document}

\maketitle

\section*{Meccanismi neurali della memoria}

\subsection*{Livelli di analisi}
\begin{itemize}
    \item \textbf{Molecolare}: Modifiche sinaptiche e plasticità
    \item \textbf{Cellulare}: Potenziamento a lungo termine (LTP)
    \item \textbf{Di circuito}: Reti neurali e engrammi
\end{itemize}

\begin{figure}[h]
    \centering
    \includegraphics[width=0.8\textwidth]{livelli_memoria} % Inserire immagine
    \caption{Livelli di analisi dei substrati neurali della memoria}
    \label{fig:livelli}
\end{figure}

\section*{Regola di Hebb e plasticità sinaptica}

\subsection*{"Neurons that fire together, wire together"}
\begin{itemize}
    \item Formulata da Donald Hebb (1949)
    \item Base teorica del potenziamento a lungo termine (LTP)
    \item Meccanismo fondamentale per l'apprendimento associativo
\end{itemize}

\begin{figure}[h]
    \centering
    \includegraphics[width=0.6\textwidth]{hebb} % Inserire immagine
    \caption{Rappresentazione grafica della regola di Hebb}
    \label{fig:hebb}
\end{figure}

\section*{Potenziamento a lungo termine (LTP)}

\subsection*{Caratteristiche fondamentali}
\begin{itemize}
    \item \textbf{Associatività}: Coinvolge neuroni attivati contemporaneamente
    \item \textbf{Specificità}: Limitato alle sinapsi attivate
    \item \textbf{Coincidenza temporale}: Richiesta simultaneità pre-post sinaptica
\end{itemize}

\subsection*{Evidenze sperimentali}
\begin{itemize}
    \item Topi KO per adenilato ciclasi: deficit di LTP e memoria
    \item Correlazione spaziale e temporale con l'apprendimento
    \item Condivisione di pathway molecolari
\end{itemize}

\begin{figure}[h]
    \centering
    \includegraphics[width=0.7\textwidth]{ltp_memoria} % Inserire immagine
    \caption{Correlazione tra LTP e performance mnestiche}
    \label{fig:ltp}
\end{figure}

\section*{Engramma e teorie della memoria}

\subsection*{Esperimenti di Lashley}
\begin{itemize}
    \item Ricerche sui labirinti nei ratti (1920)
    \item Conclusioni: Memoria distribuita nella corteccia
    \item Principio di equipotenzialità e azione di massa
\end{itemize}

\subsection*{Ipotesi di Hebb sull'engramma}
\begin{itemize}
    \item Assembly cellulari: gruppi di neuroni reciprocamente connessi
    \item Rafforzamento sinapsi tra neuroni co-attivati
    \item Attivazione parziale → richiamo completo del ricordo
\end{itemize}

\begin{figure}[h]
    \centering
    \includegraphics[width=0.8\textwidth]{engramma} % Inserire immagine
    \caption{Rappresentazione dell'engramma come assembly neuronale}
    \label{fig:engramma}
\end{figure}

\section*{Consolidamento della memoria}

\subsection*{Teorie a confronto}
\begin{table}[h]
\centering
\caption{Principali teorie del consolidamento}
\begin{tabular}{p{5cm}p{10cm}}
\toprule
\textbf{Teoria} & \textbf{Descrizione} \\
\midrule
Consolidamento sistemico & Trasferimento ippocampo → corteccia \\
Tracce multiple & Ippocampo partecipa al recupero a lungo termine \\
Ricostruzione eventi & Recupero come processo ricostruttivo \\
\bottomrule
\end{tabular}
\end{table}

\section*{Apprendimento non associativo}

\subsection*{Habituazione e sensibilizzazione in Aplysia}
\begin{itemize}
    \item Modello sperimentale di Eric Kandel (Premio Nobel 2000)
    \item \textbf{Habituazione}: Depressione sinaptica
    \item \textbf{Sensibilizzazione}: Facilitazione presinaptica
    \item Meccanismi molecolari:
    \begin{itemize}
        \item Breve termine: modificazioni proteiche
        \item Lungo termine: espressione genica e crescita sinaptica
    \end{itemize}
\end{itemize}

\begin{figure}[h]
    \centering
    \includegraphics[width=0.9\textwidth]{aplysia} % Inserire immagine
    \caption{Circuiti neurali per l'apprendimento in Aplysia}
    \label{fig:aplysia}
\end{figure}

\section*{Apprendimento associativo}

\subsection*{Condizionamento classico}
\begin{itemize}
    \item Modello di Pavlov: riflesso di salivazione
    \item Estensione: riflesso di ammiccamento
    \item Ruolo chiave del cervelletto
    \item Schema: NS → Nessuna risposta | NI → RI → Associazione → NS = NC → RC
\end{itemize}

\subsection*{Condizionamento alla paura}
\begin{itemize}
    \item Ruolo centrale dell'amigdala
    \item Caso clinico: paziente S.M. (sindrome di Urbach-Wiethe)
    \item Circuito: Talamo → Amigdala → Risposte autonomiche
\end{itemize}

\begin{figure}[h]
    \centering
    \includegraphics[width=0.7\textwidth]{amigdala} % Inserire immagine
    \caption{Circuiti dell'amigdala nel condizionamento alla paura}
    \label{fig:amigdala}
\end{figure}

\subsection*{Condizionamento operante}
\begin{itemize}
    \item Modello di Skinner: gabbia con leva
    \item Apprendimento per conseguenze (rinforzo/punizione)
    \item Estinzione: riduzione della risposta appresa
\end{itemize}

\end{document}