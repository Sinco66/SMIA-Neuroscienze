\documentclass[11pt]{article}
\usepackage[italian]{babel}
\usepackage[utf8]{inputenc}
\usepackage[T1]{fontenc}
\usepackage{geometry}
\usepackage{graphicx}
\usepackage{amsfonts}
\usepackage{amssymb}
\usepackage{booktabs}
\usepackage{multirow}
\usepackage{enumitem}
\geometry{a4paper, left=2cm, right=2cm, top=2cm, bottom=2.5cm}

\title{Relazione struttura-funzione}
\author{Orcam}
\date{}

\begin{document}

\maketitle

\section*{Dibattito storico: Localizzazionismo vs Olismo}

\subsection*{Prospettive storiche}
\begin{itemize}
    \item \textbf{Ippocrate (470-370 a.C.)}: Identificò il cervello come sede della coscienza
    \item \textbf{Aristotele (384-322 a.C.)}: Considerava il cuore centro delle funzioni vitali
\end{itemize}

\begin{figure}[h]
    \centering
    \includegraphics[width=0.8\textwidth]{ippocrate_aristotele} % Inserire immagine
    \caption{Concezioni storiche sul ruolo del cervello}
    \label{fig:storico}
\end{figure}

\subsection*{Teorie contrastanti}
\begin{table}[h]
\centering
\caption{Confronto tra localizzazionismo e olismo}
\begin{tabular}{p{7cm}p{7cm}}
\toprule
\textbf{Localizzazionismo} & \textbf{Olismo} \\
\midrule
Funzioni cerebrali specifiche localizzate in aree circoscritte & Funzioni cerebrali distribuite in tutto il cervello \\
Sostenuto da Franz J. Gall (frenologia) & Sostenuto da Pierre Flourens e Karl Lashley \\
Esempio: Aree del linguaggio di Broca e Wernicke & Principio dell'equipotenzialità corticale \\
\bottomrule
\end{tabular}
\end{table}

\section*{Evidenze per il localizzazionismo}

\subsection*{Aree del linguaggio}
\begin{itemize}
    \item \textbf{Area di Broca} (corteccia frontale sinistra):
    \begin{itemize}
        \item Afasia motoria: difficoltà nella produzione del linguaggio
        \item Comprensione conservata
        \item Caso del paziente "Tan"
    \end{itemize}
    
    \item \textbf{Area di Wernicke} (corteccia temporale sinistra):
    \begin{itemize}
        \item Afasia sensoriale: linguaggio fluente ma privo di senso
        \item Comprensione compromessa
    \end{itemize}
    
    \item \textbf{Fascicolo arcuato}:
    \begin{itemize}
        \item Afasia di conduzione: difficoltà nella ripetizione di parole
        \item Connessione tra aree di Broca e Wernicke
    \end{itemize}
\end{itemize}

\begin{figure}[h]
    \centering
    \includegraphics[width=0.9\textwidth]{aree_linguaggio} % Inserire immagine
    \caption{Aree cerebrali specializzate per il linguaggio}
    \label{fig:linguaggio}
\end{figure}

\subsection*{Altri casi clinici}
\begin{itemize}
    \item \textbf{Phineas Gage}:
    \begin{itemize}
        \item Danno al lobo frontale
        \item Alterazioni della personalità e del comportamento sociale
        \item Dimostrazione del ruolo frontale nelle funzioni esecutive
    \end{itemize}
    
    \item \textbf{Sindrome della mano aliena}:
    \begin{itemize}
        \item Danno al lobo parietotemporale
        \item Movimenti involontari dell'arto
        \item Senso di non appartenenza dell'arto
    \end{itemize}
\end{itemize}

\begin{figure}[h]
    \centering
    \includegraphics[width=0.6\textwidth]{phineas_gage} % Inserire immagine
    \caption{Il caso di Phineas Gage}
    \label{fig:gage}
\end{figure}

\section*{Mappe corticali}

\subsection*{Homunculus di Penfield}
\begin{itemize}
    \item Mappe della rappresentazione corporea nelle cortecce:
    \begin{itemize}
        \item Corteccia motoria primaria (M1)
        \item Corteccia somatosensoriale primaria (S1)
    \end{itemize}
    \item Rappresentazione sproporzionata delle parti del corpo
    \item Principio: "Quanto più fine è il controllo motorio o sensoriale, tanto maggiore è l'area corticale dedicata"
\end{itemize}

\begin{figure}[h]
    \centering
    \includegraphics[width=0.7\textwidth]{homunculus} % Inserire immagine
    \caption{Homunculus motorio e sensitivo di Penfield}
    \label{fig:homunculus}
\end{figure}

\section*{Visione moderna: reti neurali distribuite}

\begin{itemize}
    \item \textbf{Integrazione delle prospettive}:
    \begin{itemize}
        \item Specializzazione funzionale di aree specifiche
        \item Integrazione attraverso reti neurali distribuite
    \end{itemize}
    
    \item \textbf{Principi fondamentali}:
    \begin{itemize}
        \item Componenti di un comportamento elaborati in zone diverse
        \item Una regione può partecipare a più funzioni
        \item Connettività funzionale dinamica e adattiva
    \end{itemize}
    
    \item \textbf{Esempio}: 
    \begin{itemize}
        \item Comprensione del linguaggio coinvolge:
        \item Corteccia uditiva primaria $\rightarrow$ Area di Wernicke $\rightarrow$ Fascicolo arcuato $\rightarrow$ Area di Broca
    \end{itemize}
\end{itemize}

\begin{figure}[h]
    \centering
    \includegraphics[width=0.8\textwidth]{reti_neurali} % Inserire immagine
    \caption{Reti neurali distribuite per funzioni complesse}
    \label{fig:reti}
\end{figure}

\end{document}