\documentclass[11pt]{article}
\usepackage[italian]{babel}
\usepackage[utf8]{inputenc}
\usepackage[T1]{fontenc}
\usepackage{geometry}
\usepackage{graphicx}
\usepackage{amsmath}
\usepackage{amssymb}
\usepackage{booktabs}
\usepackage{multirow}
\usepackage{enumitem}
\geometry{a4paper, left=2cm, right=2cm, top=2cm, bottom=2.5cm}

\title{Memoria I}
\author{Orcam}
\date{}

\begin{document}

\maketitle

\section*{Apprendimento e memoria}

\begin{itemize}
    \item \textbf{Apprendimento}: Acquisizione di nuove conoscenze o abilità
    \item \textbf{Memoria}: Conservazione delle informazioni apprese
    \item Adattamenti dei circuiti cerebrali all'ambiente
\end{itemize}

\section*{Caso clinico: Henry Molaison (H.M.)}

\subsection*{Caratteristiche della lesione}
\begin{itemize}
    \item Lobectomia bilaterale della corteccia temporale mediale
    \item Rimozione di ippocampo e amigdala
    \item Conseguenze:
    \begin{itemize}
        \item Amnesia anterograda: incapacità di formare nuovi ricordi
        \item Amnesia retrograda: perdita di ricordi pre-operatori
        \item Memoria procedurale intatta
    \end{itemize}
\end{itemize}

\begin{figure}[h]
    \centering
    \includegraphics[width=0.7\textwidth]{paziente_HM} % Inserire immagine
    \caption{Localizzazione della lesione nel paziente H.M.}
    \label{fig:hm}
\end{figure}

\section*{Stadi della memoria}

\begin{enumerate}
    \item \textbf{Codifica}: Elaborazione iniziale delle informazioni
    \item \textbf{Consolidamento}: Stabilizzazione del ricordo
    \item \textbf{Immagazzinamento}: Conservazione a lungo termine
    \item \textbf{Recupero}: Accesso alle informazioni memorizzate
\end{enumerate}

\section*{Classificazione della memoria}

\subsection*{Per durata}
\begin{table}[h]
\centering
\caption{Tipi di memoria per durata}
\begin{tabular}{p{5cm}p{10cm}}
\toprule
\textbf{Tipo} & \textbf{Caratteristiche} \\
\midrule
Memoria sensoriale & Transitoria (max 1000 ms), elevata capacità \\
Memoria a breve termine & Durata limitata (sec/min), capacità 7±2 elementi \\
Memoria di lavoro & Manipolazione attiva delle informazioni \\
Memoria a lungo termine & Conservazione duratura (ore/anni), capacità elevata \\
\bottomrule
\end{tabular}
\end{table}

\subsection*{Per contenuto}
\begin{figure}[h]
    \centering
    \includegraphics[width=0.9\textwidth]{tipi_memoria} % Inserire immagine
    \caption{Classificazione della memoria dichiarativa e non dichiarativa}
    \label{fig:tipi_memoria}
\end{figure}

\begin{itemize}
    \item \textbf{Memoria dichiarativa (esplicita)}:
    \begin{itemize}
        \item Episodica (eventi autobiografici)
        \item Semantica (fatti e conoscenze generali)
        \item Dipendente dal lobo temporale mediale
    \end{itemize}
    
    \item \textbf{Memoria non dichiarativa (implicita)}:
    \begin{itemize}
        \item Procedurale (abilità motorie)
        \item Priming (attivazione percettiva)
        \item Condizionamento classico
        \item Apprendimento non associativo
    \end{itemize}
\end{itemize}

\section*{Dimenticanza}

\subsection*{Curva dell'oblio di Ebbinghaus}
\begin{itemize}
    \item Perdita esponenziale delle informazioni nel tempo
    \item 58\% perso dopo 20 minuti
    \item 75\% perso dopo 24 ore
\end{itemize}

\begin{figure}[h]
    \centering
    \includegraphics[width=0.6\textwidth]{curva_oblio} % Inserire immagine
    \caption{Curva dell'oblio di Ebbinghaus}
    \label{fig:ebbinghaus}
\end{figure}

\subsection*{Teorie della dimenticanza}
\begin{enumerate}
    \item Decadimento (traccia mnestica si indebolisce)
    \item Interferenza (nuove informazioni sovrascrivono le vecchie)
    \item Disuso (mancanza di recupero)
\end{enumerate}

\section*{Amnesia}

\begin{table}[h]
\centering
\caption{Confronto tra amnesia anterograda e retrograda}
\begin{tabular}{p{7cm}p{7cm}}
\toprule
\textbf{Amnesia anterograda} & \textbf{Amnesia retrograda} \\
\midrule
Incapacità di formare nuovi ricordi & Perdita di ricordi pre-esistenti \\
Danni all'ippocampo & Danni a corteccia associativa \\
Esempio: paziente H.M. & Esempio: sindrome di Korsakoff \\
\bottomrule
\end{tabular}
\end{table}

\section*{Sistemi neurali della memoria}

\subsection*{Memoria dichiarativa}
\begin{itemize}
    \item Circuito ippocampale:
    \begin{itemize}
        \item Ippocampo
        \item Corteccia entorinale
        \item Corteccia peririnale
        \item Corteccia paraippocampale
    \end{itemize}
    \item Strutture diencefaliche:
    \begin{itemize}
        \item Corpi mammillari
        \item Talamo dorsomediale
    \end{itemize}
\end{itemize}

\begin{figure}[h]
    \centering
    \includegraphics[width=0.8\textwidth]{circuito_ippocampale} % Inserire immagine
    \caption{Circuiti neurali per la memoria dichiarativa}
    \label{fig:circuito}
\end{figure}

\subsection*{Plasticità ippocampale}
\begin{itemize}
    \item \textbf{Place cells}: Neuroni che si attivano in posizioni specifiche (John O'Keefe)
    \item \textbf{Grid cells}: Neuroni con attivazione a griglia esagonale (Moser e Moser)
    \item \textbf{Head direction cells}: Codificano la direzione della testa
\end{itemize}

\begin{figure}[h]
    \centering
    \includegraphics[width=0.7\textwidth]{place_cells} % Inserire immagine
    \caption{Place cells nell'ippocampo}
    \label{fig:place_cells}
\end{figure}

\subsection*{Evidenze sperimentali}
\begin{itemize}
    \item \textbf{Tassisti di Londra}: Ippocampo posteriore più sviluppato (Maguire et al.)
    \item \textbf{Morris Water Maze}: Deficit spaziali dopo lesione ippocampale
    \item \textbf{fMRI}: Attivazione ippocampale durante compiti di memoria
\end{itemize}

\end{document}