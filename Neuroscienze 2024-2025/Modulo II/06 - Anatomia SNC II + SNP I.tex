\documentclass[11pt]{article}
\usepackage[italian]{babel}
\usepackage[utf8]{inputenc}
\usepackage[T1]{fontenc}
\usepackage{geometry}
\usepackage{graphicx}
\usepackage{amsmath}
\usepackage{amssymb}
\usepackage{enumitem}
\usepackage{booktabs}
\usepackage{multirow}
\geometry{a4paper, left=2cm, right=2cm, top=2cm, bottom=2.5cm}

\title{Anatomia SNC II}
\author{Orcam}
\date{}

\begin{document}

\maketitle

\section*{Sistemi di protezione del SNC}

\subsection*{Strutture ossee}
\begin{itemize}
    \item Scatola cranica (protegge l'encefalo)
    \item Colonna vertebrale (protegge il midollo spinale)
    \item Forame magno (passaggio tra encefalo e midollo spinale)
\end{itemize}
\begin{figure}[h]
    \centering
    \includegraphics[width=0.6\textwidth]{ossa_craniche} % Inserire immagine
    \caption{Ossa craniche e forame magno}
    \label{fig:ossa}
\end{figure}

\subsection*{Meningi}
Strutture connettivali a tre strati:
\begin{enumerate}
    \item \textbf{Dura madre}: Doppio strato (encefalo), singolo strato (midollo), ricca di collagene
    \item \textbf{Aracnoide}: Priva di collagene, delimita spazio subaracnoideo
    \item \textbf{Pia madre}: Aderente al tessuto nervoso
\end{enumerate}
\begin{figure}[h]
    \centering
    \includegraphics[width=0.7\textwidth]{meningi} % Inserire immagine
    \caption{Struttura delle meningi cerebrali e spinali}
    \label{fig:meningi}
\end{figure}

\subsection*{Liquido Cefalorachidiano (CSF)}
\begin{itemize}
    \item Prodotto dai plessi coroidei (500 ml/giorno)
    \item Volume totale: 90-140 ml
    \item Composizione: basso contenuto proteico, alto Na+, basso K+
    \item Funzioni:
    \begin{itemize}
        \item Protezione meccanica
        \item Scambio di sostanze
        \item Rimozione metaboliti
        \item Stabilità ionica
    \end{itemize}
    \item Flusso: Ventricoli $\rightarrow$ spazio subaracnoideo $\rightarrow$ villi aracnoidei $\rightarrow$ seni venosi
\end{itemize}
\begin{figure}[h]
    \centering
    \includegraphics[width=0.8\textwidth]{plesso_coroideo} % Inserire immagine
    \caption{Plesso coroideo e flusso del CSF}
    \label{fig:csf}
\end{figure}

\subsection*{Barriera Emato-Encefalica (BBB)}
\begin{itemize}
    \item Struttura specializzata dei capillari cerebrali
    \item Componenti principali:
    \begin{itemize}
        \item Cellule endoteliali con tight junctions
        \item Periciti
        \item Astrociti
        \item Membrana basale
    \end{itemize}
    \item Funzione: selettività per sostanze
\end{itemize}
\begin{figure}[h]
    \centering
    \includegraphics[width=0.7\textwidth]{barriera_encefalica} % Inserire immagine
    \caption{Struttura della barriera emato-encefalica}
    \label{fig:bbb}
\end{figure}

\section*{Organizzazione del Sistema Nervoso Periferico}

\subsection*{Divisioni principali}
\begin{table}[h]
\centering
\caption{Classificazione del SNP}
\begin{tabular}{p{6cm}p{8cm}}
\toprule
\textbf{Sistema somatico} & \textbf{Sistema autonomo} \\
\midrule
Innerva muscoli scheletrici, organi di senso e cute & Innerva muscoli lisci, muscolo cardiaco e ghiandole \\
Controllo volontario & Controllo involontario \\
Afferenze da cute, muscoli, articolazioni & Afferenze da organi interni \\
Efferenze motorie somatiche & Efferenze motorie viscerali (simpatico, parasimpatico, enterico) \\
\bottomrule
\end{tabular}
\end{table}

\subsection*{Nervi cranici}
12 paia con funzioni specifiche:
\begin{table}[h]
\centering
\caption{Classificazione nervi cranici}
\begin{tabular}{lll}
\toprule
\textbf{Tipo} & \textbf{Nervi} & \textbf{Funzione principale} \\
\midrule
Sensitivi & I (olfattivo), II (ottico), VIII (vestibolococleare) & Olfatto, vista, udito/equilibrio \\
Motori & III (oculomotore), IV (trocleare), VI (abducente), XI (accessorio), XII (ipoglosso) & Movimento occhi, collo, lingua \\
Misti & V (trigemino), VII (facciale), IX (glossofaringeo), X (vago) & Sensibilità/motilità facciale, deglutizione, innervazione viscerale \\
\bottomrule
\end{tabular}
\end{table}
\begin{figure}[h]
    \centering
    \includegraphics[width=0.9\textwidth]{nervi_cranici} % Inserire immagine
    \caption{Decorso dei nervi cranici}
    \label{fig:nervi_cranici}
\end{figure}

\subsection*{Nervi spinali}
31 paia organizzate in:
\begin{itemize}
    \item 8 cervicali (C1-C8)
    \item 12 toracici (T1-T12)
    \item 5 lombari (L1-L5)
    \item 5 sacrali (S1-S5)
    \item 1 coccigeo (Co)
\end{itemize}
Struttura di un nervo spinale:
\begin{itemize}
    \item Radice dorsale (sensitiva)
    \item Radice ventrale (motoria)
    \item Ganglio della radice dorsale (corpi cellulari neuroni sensitivi)
\end{itemize}
\begin{figure}[h]
    \centering
    \includegraphics[width=0.7\textwidth]{nervi_spinali} % Inserire immagine
    \caption{Struttura dei nervi spinali}
    \label{fig:nervi_spinali}
\end{figure}

\subsection*{Sensibilità}
\begin{itemize}
    \item \textbf{Somatica}:
    \begin{itemize}
        \item Esterocettiva (tatto, temperatura, dolore)
        \item Propriocettiva (posizione corporea)
    \end{itemize}
    \item \textbf{Speciale}: vista, udito, olfatto, gusto
    \item \textbf{Vestibolare}: posizione/movimento testa
    \item \textbf{Viscerale}: segnali da organi interni
\end{itemize}

\subsection*{Sistema Nervoso Autonomo}
Divisione efferente viscerale:
\begin{table}[h]
\centering
\caption{Confronto sistema simpatico e parasimpatico}
\begin{tabular}{p{7cm}p{7cm}}
\toprule
\textbf{Simpatico} & \textbf{Parasimpatico} \\
\midrule
Attivazione ("lotta o fuga") & Conservazione energia ("riposo e digestione") \\
Origine: midollo spinale toraco-lombare & Origine: tronco encefalico e midollo sacrale \\
Gangli paravertebrali e prevertebrali & Gangli vicino agli organi bersaglio \\
Neurotrasmettitore: noradrenalina & Neurotrasmettitore: acetilcolina \\
\bottomrule
\end{tabular}
\end{table}
\begin{figure}[h]
    \centering
    \includegraphics[width=0.9\textwidth]{sistema_autonomo} % Inserire immagine
    \caption{Organizzazione del sistema nervoso autonomo}
    \label{fig:autonomo}
\end{figure}

\subsection*{Sistema Nervoso Enterico}
Rete neuronale autonoma per il tratto gastrointestinale:
\begin{itemize}
    \item \textbf{Plesso di Auerbach}: controllo motilità muscolare
    \item \textbf{Plesso di Meissner}: controllo secrezioni ghiandolari
    \item Integrazione con sistema autonomo (simpatico e parasimpatico)
\end{itemize}
\begin{figure}[h]
    \centering
    \includegraphics[width=0.8\textwidth]{sistema_enterico} % Inserire immagine
    \caption{Innervazione del tratto gastrointestinale}
    \label{fig:enterico}
\end{figure}

\end{document}