\documentclass{article}
\usepackage[italian]{babel}
\usepackage{graphicx}
\usepackage{amsmath}
\usepackage{booktabs}
\usepackage{multirow}
\usepackage{siunitx}

\title{Proprietà attive delle membrane cellulari}
\author{Orcam}
\date{}

\begin{document}

\maketitle

% Sezione 1: Proprietà passive vs attive
\section{Proprietà elettriche delle membrane}
\begin{itemize}
\item \textbf{Passive}:
  \begin{itemize}
  \item Capacità (\(C_m \approx \SI{1}{\micro\farad\per\centi\meter\squared}\))
  \item Resistenza (\(R_m\))
  \item Risposta lineare (legge di Ohm)
  \end{itemize}

\item \textbf{Attive}:
  \begin{itemize}
  \item Canali ionici voltaggio-dipendenti
  \item Generazione di potenziali d'azione
  \item Risposta non lineare
  \end{itemize}

% Sezione 2: Potenziali elettrotonici
\section{Potenziali elettrotonici}
\begin{itemize}
\item Risposta passiva a correnti sottosoglia:
\[
V(t) = V_{\infty} \left(1 - e^{-t/\tau}\right) \quad \text{con} \quad \tau = R_m C_m
\]
\begin{itemize}
\item \(\tau\): Costante di tempo (tipicamente \SIrange{1}{20}{\milli\second})
\end{itemize}

\begin{figure}[h]
\centering
\includegraphics[width=0.5\textwidth]{elettrotonico.png}
\caption{Carica del condensatore di membrana}
\label{fig:elettrotonico}
\end{figure}
\end{itemize}

% Sezione 3: Legge di Ohm nelle membrane
\section{Relazione corrente-voltaggio}
\[
\Delta V_m = I_m R_m = \frac{I_m}{g_m}
\]
\begin{itemize}
\item Linearità mantenuta solo in condizioni passive
\item Deviazione durante attivazione canali voltaggio-dipendenti

\begin{figure}[h]
\centering
\includegraphics[width=0.5\textwidth]{ohm_curve.png}
\caption{Curva I-V in condizioni passive e attive}
\label{fig:ohm}
\end{figure}
\end{itemize}

% Sezione 4: Potenziale d'azione
\section{Il potenziale d'azione}
\begin{itemize}
\item Caratteristiche chiave:
  \begin{itemize}
  \item Soglia di attivazione (\(\approx \SI{-55}{\milli\volt}\))
  \item Fase ascendente (apertura canali Na\(^+\))
  \item Fase discendente (inattivazione Na\(^+\) + apertura K\(^+\))
  \item Iperpolarizzazione postuma
  \end{itemize}

\item Proprietà:
  \begin{itemize}
  \item Legge del "tutto o nulla"
  \item Inversione del potenziale (\(V_m \approx +\SI{60}{\milli\volt}\))
  \item Durata: \(\sim\SI{1}{\milli\second}\) (neuroni)
  \end{itemize}

\begin{figure}[h]
\centering
\includegraphics[width=0.7\textwidth]{potenziale_azione.png}
\caption{Fasi del potenziale d'azione}
\label{fig:azione}
\end{figure}
\end{itemize}

% Sezione 5: Canali ionici voltaggio-dipendenti
\section{Canali voltaggio-dipendenti}
\begin{itemize}
\item Meccanismo di attivazione:
  \begin{itemize}
  \item Sensori di voltaggio (domini S4)
  \item Apertura rapida (Na\(^+\)) vs lenta (K\(^+\))
  \end{itemize}

\item Equazione della conduttanza:
\[
g_{Na} = \bar{g}_{Na} \cdot m^3 h \quad;\quad g_K = \bar{g}_K \cdot n^4
\]
\begin{itemize}
\item \(m, h, n\): Variabili di attivazione/inattivazione
\end{itemize}

\begin{figure}[h]
\centering
\includegraphics[width=0.6\textwidth]{canali_na_k.png}
\caption{Dinamica dei canali Na\(^+\) e K\(^+\)}
\label{fig:canali}
\end{figure}
\end{itemize}

% Sezione 6: Metodi di studio
\section{Tecniche sperimentali}
\begin{itemize}
\item \textbf{Current clamp}:
  \begin{itemize}
  \item Misura \(V_m\) in risposta a correnti iniettate
  \item Ideale per studiare potenziali d'azione
  \end{itemize}

\item \textbf{Voltage clamp}:
  \begin{itemize}
  \item Mantiene \(V_m\) costante
  \item Misura correnti ioniche (\(I = g \cdot \Delta V\))
  \end{itemize}

\begin{figure}[h]
\centering
\includegraphics[width=0.5\textwidth]{voltage_clamp.png}
\caption{Schema del sistema voltage clamp}
\label{fig:clamp}
\end{figure}
\end{itemize}

% Sezione 7: Modello sperimentale
\section{Assone gigante di calamaro}
\begin{itemize}
\item Caratteristiche: \begin{itemize}
  \item Diametro: \(\sim\SI{800}{\micro\meter}\)
  \item Facilità di manipolazione sperimentale
  \end{itemize}

\item Contributi chiave:
  \begin{itemize}
  \item Scoperta dei canali Na\(^+\)/K\(^+\) voltaggio-dipendenti
  \item Modello per lo studio della conduzione assonica
  \end{itemize}

\begin{figure}[h]
\centering
\includegraphics[width=0.4\textwidth]{assone_calamaro.png}
\caption{Assone gigante e sistema di registrazione}
\label{fig:calamaro}
\end{figure}
\end{itemize}

\end{document}