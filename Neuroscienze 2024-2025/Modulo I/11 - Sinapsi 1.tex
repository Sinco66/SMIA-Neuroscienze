\documentclass{article}
\usepackage[italian]{babel}
\usepackage{graphicx}
\usepackage{booktabs}
\usepackage{amsmath}
\usepackage{hyperref}
\usepackage[margin=1in]{geometry}

\title{Sinapsi I}
\author{Orcam}
\date{}

\begin{document}

\maketitle

\section{Segnali Nervosi}
\subsection{Potenziali Elettrotonici vs Potenziali d'Azione}
\begin{table}[h]
\centering
\caption{Confronto tra potenziali elettrotonici e potenziali d'azione}
\begin{tabular}{p{6cm}p{6cm}}
\toprule
\textbf{Potenziali Elettrotonici} & \textbf{Potenziali d'Azione} \\
\midrule
• Ampiezza graduata (dipende dall'intensità dello stimolo) & • Fenomeno "tutto o nulla" \\
• Decadono con la distanza & • Si autorigenerano (mantengono ampiezza) \\
• Si sommano temporalmente/spazialmente & • Non sommabili (periodo refrattario) \\
• Seguono la legge di Ohm & • Uguali in ampiezza per stesso neurone \\
& • Codifica intensità nella frequenza \\
\bottomrule
\end{tabular}
\label{tab:potenziali}
\end{table}

% Spazio per Figura 1: Propagazione segnale nervoso
\begin{figure}[h]
\centering
\includegraphics[width=0.8\textwidth]{figura1_propagazione}
\caption{Alternanza di potenziali elettrotonici e d'azione nella trasmissione del segnale}
\label{fig:propagazione}
\end{figure}

\section{Reti Neurali}
\subsection{Complessità delle Reti}
\begin{itemize}
\item Circuiti neurali con diverse complessità: cerebellari, spinali, corticali
\item Modelli di integrazione spaziale non lineare nei dendriti
\end{itemize}

\subsection{Modelli Computazionali}
\begin{itemize}
\item \textbf{Perceptron (singolo strato)}:
  \begin{equation*}
    f(x) = \begin{cases} 
      1 & \text{se } \sum w_i x_i + b > 0 \\
      0 & \text{se } \sum w_i x_i + b < 0 
    \end{cases}
  \end{equation*}
\item \textbf{Multi-Layer Perceptron}: 
  \begin{itemize}
  \item Strati nascosti per estrarre features gerarchiche
  \item Esempio: riconoscimento di volti (pixel $\rightarrow$ bordi $\rightarrow$ features complesse)
  \end{itemize}
\end{itemize}

% Spazio per Figura 2: Modelli perceptron
\begin{figure}[h]
\centering
\includegraphics[width=0.9\textwidth]{figura2_perceptron}
\caption{Architettura di perceptron singolo e multi-strato}
\label{fig:perceptron}
\end{figure}

\section{Storia delle Sinapsi}
\subsection{Teorie a Confronto}
\begin{itemize}
\item \textbf{Teoria Reticolare} (Golgi): cellule nervose unite da ponti protoplasmatici
\item \textbf{Teoria Neuronale} (Cajal): cellule indipendenti che comunicano tramite giunzioni specializzate
\item Conferma della teoria neuronale con la colorazione di Golgi-Cajal
\end{itemize}

% Spazio per Figura 3: Teorie sinaptiche
\begin{figure}[h]
\centering
\includegraphics[width=0.7\textwidth]{figura3_teorie}
\caption{Confronto tra teoria reticolare e teoria neuronale}
\label{fig:teorie}
\end{figure}

\section{Proprietà delle Sinapsi}
\subsection{Convergenza e Divergenza}
\begin{itemize}
\item \textbf{Convergenza}: Molti neuroni presinaptici $\rightarrow$ un neurone postsinaptico
\item \textbf{Divergenza}: Un neurone presinaptico $\rightarrow$ molti neuroni postsinaptici
\end{itemize}

\subsection{Tipi di Sinapsi}
\begin{itemize}
\item \textbf{Asso-somatiche}: Tra assone e corpo cellulare
\item \textbf{Asso-dendritiche}: Tra assone e dendriti
\item \textbf{Asso-assoniche}: Tra assoni (modulazione)
\end{itemize}

% Spazio per Figura 4: Tipi di sinapsi
\begin{figure}[h]
\centering
\includegraphics[width=0.8\textwidth]{figura4_tipi_sinapsi}
\caption{Localizzazione e tipi di sinapsi}
\label{fig:tipi_sinapsi}
\end{figure}

\section{Sinapsi Elettriche}
\subsection{Struttura e Funzione}
\begin{itemize}
\item \textbf{Gap junctions}: Canali idrofilici formati da connessine
\item \textbf{Meccanismo}: Passaggio diretto di ioni e piccole molecole (<500 Da)
\item \textbf{Vantaggi}: Velocità, sincronizzazione
\item \textbf{Svantaggi}: Dispersione di carica, bidirezionalità
\end{itemize}

% Spazio per Figura 5: Sinapsi elettriche
\begin{figure}[h]
\centering
\includegraphics[width=0.6\textwidth]{figura5_sinapsi_elettriche}
\caption{Struttura delle gap junctions e trasmissione elettrica}
\label{fig:elettriche}
\end{figure}

\section{Sinapsi Chimiche}
\subsection{Struttura}
\begin{itemize}
\item \textbf{Elementi chiave}:
  \begin{itemize}
  \item Vescicole sinaptiche
  \ta Fessura sinaptica (20nm)
  \item Recettori postsinaptici
  \end{itemize}
\item \textbf{Tempi}: Risposta postsinaptica $\sim$1 ms
\end{itemize}

\subsection{Neurotrasmettitori}
\begin{table}[h]
\centering
\caption{Principali neurotrasmettitori e funzioni}
\begin{tabular}{ll}
\toprule
\textbf{Neurotrasmettitore} & \textbf{Funzione} \\
\midrule
Glutammato & Memoria, eccitazione \\
GABA & Inibizione, calma \\
Dopamina & Piacere, motivazione \\
Serotonina & Umore, sonno \\
Acetilcolina & Apprendimento, contrazione muscolare \\
Endorfine & Analgesia, euforia \\
\bottomrule
\end{tabular}
\label{tab:neurotrasmettitori}
\end{table}

% Spazio per Figura 6: Sinapsi chimiche
\begin{figure}[h]
\centering
\includegraphics[width=0.7\textwidth]{figura6_sinapsi_chimiche}
\caption{Meccanismo di trasmissione chimica}
\label{fig:chimiche}
\end{figure}

\section{Rilascio dei Neurotrasmettitori}
\subsection{Fasi del Ciclo Sinaptico}
\begin{enumerate}
\item \textbf{Sintesi}:
  \begin{itemize}
  \item Neurotrasmettitori a basso PM: sintesi nel bottone sinaptico
  \item Neuropeptidi: sintesi nel soma, trasporto assonale
  \end{itemize}
  
\item \textbf{Ancoraggio e priming}: 
  \begin{itemize}
  \item Interazione proteine SNARE (v-SNARE/t-SNARE)
  \item Ruolo della sinaptotagmina (sensore Ca$^{2+}$)
  \end{itemize}

\item \textbf{Rilascio}:
  \begin{itemize}
  \item Apertura canali Ca$^{2+}$ voltaggio-dipendenti
  \item Fusione vescicolare mediata da Ca$^{2+}$
  \end{itemize}

\item \textbf{Recupero}:
  \begin{itemize}
  \item Endocitosi e riciclo vescicolare (30s-1min)
  \item Ruolo della dinamina
  \end{itemize}
\end{enumerate}

% Spazio per Figura 7: Ciclo sinaptico
\begin{figure}[h]
\centering
\includegraphics[width=0.9\textwidth]{figura7_ciclo_sinaptico}
\caption{Fasi del ciclo di rilascio e recupero delle vescicole}
\label{fig:ciclo}
\end{figure}

\subsection{Mobilizzazione Vescicole}
\begin{itemize}
\item Fosforilazione della sinapsina da parte di chinasi Ca$^{2+}$-dipendenti
\item Rilascio delle vescicole dal citoscheletro di actina
\item Migrazione verso la membrana presinaptica
\end{itemize}

% Spazio per Figura 8: Mobilizzazione vescicole
\begin{figure}[h]
\centering
\includegraphics[width=0.8\textwidth]{figura8_mobilizzazione}
\caption{Meccanismo di mobilizzazione delle vescicole di riserva}
\label{fig:mobilizzazione}
\end{figure}

\end{document}