\documentclass{article}
\usepackage[italian]{babel}
\usepackage{graphicx}
\usepackage{amsmath}
\usepackage{booktabs}
\usepackage{multirow}
\usepackage{siunitx}

\title{Potenziale di equilibrio e Proprietà passive delle membrane cellulari}
\author{Orcam}
\date{}

\begin{document}

\maketitle

% Sezione 1: Potenziale di equilibrio e Equazione di Nernst
\section{Potenziale di equilibrio e Equazione di Nernst}
\begin{itemize}
\item Il potenziale di equilibrio (\(E_x\)) per uno ione è determinato dall'equazione di Nernst:
\[
E_x = \frac{RT}{zF} \ln \frac{[X]_e}{[X]_i} \quad \text{o, a 38°C:} \quad E_x = 0.061 \log \frac{[X]_e}{[X]_i}
\]
\begin{table}[h]
\centering
\begin{tabular}{lcc}
\toprule
Ione & \(E_x\) teorico & Concentrazioni (mM) \\
\midrule
K\(^+\) & -94 mV & 140 (int) / 4 (ext) \\
Na\(^+\) & +65 mV & 10 (int) / 120 (ext) \\
Cl\(^-\) & -90 mV & 4 (int) / 120 (ext) \\
\bottomrule
\end{tabular}
\caption{Potenziali di equilibrio per i principali ioni}
\end{table}

\begin{figure}[h]
\centering
\includegraphics[width=1\textwidth]{Neuroscienze 2024-2025/Modulo I/Immagini Modulo I/Screenshot 2025-06-21 at 17-51-15 7. Potenziale di equilibrio_Proprietà passive membrane cellulari .pdf.png}
\caption{Deviazione dalla previsione di Nernst a basse [K\(^+\)] esterne}
\label{fig:nernst}
\end{figure}
\end{itemize}

% Sezione 2: Equazione di Goldman-Hodgkin-Katz
\section{Equazione di Goldman-Hodgkin-Katz (GHK)}
\[
V_m = \frac{RT}{F} \ln \frac{P_K[K^+]_e + P_{Na}[Na^+]_e + P_{Cl}[Cl^-]_i}{P_K[K^+]_i + P_{Na}[Na^+]_i + P_{Cl}[Cl^-]_e}
\]
\begin{itemize}
\item Esempio con \(P_{Na} = 0.01 P_K\):
\[
V_m = 0.061 \log \frac{4 + 1.2}{140 + 0.1} \approx -96 \, \text{mV}
\]

\item Permeabilità relative:
\begin{itemize}
\item \(P_K : P_{Na} : P_{Cl} = 1 : 0.01 : 0.5\)
\end{itemize}

\begin{figure}[h]
\centering
\includegraphics[width=1\textwidth]{Neuroscienze 2024-2025/Modulo I/Immagini Modulo I/Screenshot 2025-06-21 at 17-52-06 7. Potenziale di equilibrio_Proprietà passive membrane cellulari .pdf.png}
\caption{Confronto tra equazioni di Nernst e GHK}
\label{fig:goldman}
\end{figure}
\end{itemize}

% Sezione 3: Circuito equivalente della membrana
\section{Proprietà passive: Circuito equivalente}
\begin{itemize}
\item Modello elettrico:
\begin{itemize}
\item \textbf{Capacità} (\(C_m \approx \SI{1}{\micro\farad\per\centi\meter\squared}\)): Doppio strato lipidico
\item \textbf{Resistenze} (\(R\)): Canali ionici
\end{itemize}

\[
\tau = R_m C_m \quad \text{(Costante di tempo, tipicamente 1-20 ms)}
\]

\begin{figure}[h]
\centering
\includegraphics[width=1\textwidth]{Neuroscienze 2024-2025/Modulo I/Immagini Modulo I/Screenshot 2025-06-21 at 17-52-34 7. Potenziale di equilibrio_Proprietà passive membrane cellulari .pdf.png}
\caption{Circuito equivalente: resistenze e capacità in parallelo}
\label{fig:circuito}
\end{figure}
\end{itemize}

% Sezione 4: Potenziali elettrotonici
\section{Potenziali elettrotonici}
\begin{itemize}
\item Risposta passiva a correnti iniettate:
\[
V(t) = V_{\infty} \left(1 - e^{-t/\tau}\right)
\]
\begin{itemize}
\item \(V_{\infty}\): Potenziale a regime
\item \(\tau\): Tempo per raggiungere il 63\% di \(V_{\infty}\)
\end{itemize}

\item Proprietà:
\begin{itemize}
\item Decadimento esponenziale con la distanza
\item Somma temporale/spaziale
\end{itemize}

\begin{figure}[h]
\centering
\includegraphics[width=1\textwidth]{Neuroscienze 2024-2025/Modulo I/Immagini Modulo I/Screenshot 2025-06-21 at 19-17-06 7. Potenziale di equilibrio_Proprietà passive membrane cellulari .pdf.png}
\caption{Risposta della membrana a stimoli di corrente}
\label{fig:elettrotonico}
\end{figure}
\end{itemize}

% Sezione 5: Legge di Ohm e risposta passiva
\section{Legge di Ohm nelle membrane}
\[
\Delta V_m = I_m R_m \quad \text{con} \quad R_m = \frac{1}{g_{tot}}
\]
\begin{itemize}
\item Relazione lineare corrente-voltaggio:
\begin{figure}[h]
\centering
\includegraphics[width=1\textwidth]{Neuroscienze 2024-2025/Modulo I/Immagini Modulo I/Screenshot 2025-06-21 at 19-18-21 7. Potenziale di equilibrio_Proprietà passive membrane cellulari .pdf.png}
\caption{Curva I-V per membrane passive}
\label{fig:ohm}
\end{figure}

\item Parametri tipici:
\begin{itemize}
\item Resistenza specifica: \(\SI{1000}{\ohm\centi\meter\squared}\)
\item Conduttanza \(K^+\): \(\SI{10}{\milli\siemens\per\centi\meter\squared}\)
\end{itemize}
\end{itemize}

% Sezione 6: Ruolo della pompa Na+/K+
\section{Ruolo della pompa Na\(^+\)/K\(^+\) ATPasi}
\begin{itemize}
\item Mantiene i gradienti ionici:
\begin{itemize}
\item 3 Na\(^+\) espulsi / 2 K\(^+\) introdotti per ATP
\end{itemize}

\item Contribuisce al \(\sim 10\%\) del \(V_m\) tramite:
\begin{itemize}
\item Elettrogenicità diretta
\item Mantenimento gradienti per diffusione passiva
\end{itemize}

\begin{figure}[h]
\centering
\includegraphics[width=1\textwidth]{Neuroscienze 2024-2025/Modulo I/Immagini Modulo I/Screenshot 2025-06-21 at 19-20-34 7. Potenziale di equilibrio_Proprietà passive membrane cellulari .pdf.png}
\caption{Relazione ohmica tra voltaggio e corrente}
\label{fig:pompa}
\end{figure}
\end{itemize}

\end{document}