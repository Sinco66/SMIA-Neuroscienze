\documentclass{article}
\usepackage[italian]{babel}
\usepackage{graphicx}
\usepackage{amsmath}
\usepackage{booktabs}
\usepackage{siunitx}

\title{Potenziale d'azione}
\author{Orcam}
\date{}

\begin{document}

\maketitle

% Sezione 1: Tecnica del Voltage Clamp
\section{Tecnica del Voltage Clamp}
\begin{itemize}
\item Permette di misurare correnti ioniche mantenendo \( V_m \) costante:
\[
I = g \cdot (V_m - E_x) \quad \text{(Legge di Ohm generalizzata)}
\]

\item Componenti chiave:
  \begin{itemize}
  \item Amplificatore di confronto
  \item Elettrodo di corrente
  \item Potenziale di comando (\( V_c \))
  \end{itemize}

\begin{figure}[h]
\centering
\includegraphics[width=0.7\textwidth]{voltage_clamp.png}
\caption{Schema del sistema voltage clamp}
\label{fig:clamp}
\end{figure}
\end{itemize}

% Sezione 2: Basi ioniche del potenziale d'azione
\section{Basi ioniche del potenziale d'azione}
\begin{itemize}
\item Dinamiche delle correnti:
  \begin{itemize}
  \item \textbf{Corrente precoce entrante}: mediata da Na\(^+\) (\( I_{Na} \))
  \item \textbf{Corrente tardiva uscente}: mediata da K\(^+\) (\( I_K \))
  \end{itemize}

\item Equazioni delle correnti:
\[
I_{Na} = g_{Na} \cdot (V_m - E_{Na}) \quad;\quad I_K = g_K \cdot (V_m - E_K)
\]

\begin{figure}[h]
\centering
\includegraphics[width=0.6\textwidth]{correnti_ioniche.png}
\caption{Andamento temporale delle correnti Na\(^+\) e K\(^+\)}
\label{fig:correnti}
\end{figure}
\end{itemize}

% Sezione 3: Canali voltaggio-dipendenti
\section{Canali voltaggio-dipendenti}
\subsection{Canale del Na\(^+\)}
\begin{itemize}
\item Struttura:
  \begin{itemize}
  \item 4 domini (I-IV) con sensori di voltaggio (S4)
  \item Cancello di inattivazione (ansa tra domini III e IV)
  \end{itemize}

\item Stati conformazionali:
  \begin{enumerate}
  \item Chiuso (a riposo)
  \item Aperto (durante depolarizzazione)
  \item Inattivo (post-apertura)
  \end{enumerate}

\begin{figure}[h]
\centering
\includegraphics[width=0.5\textwidth]{canale_na.png}
\caption{Struttura e stati del canale Na\(^+\)}
\label{fig:canale_na}
\end{figure}
\end{itemize}

\subsection{Canale del K\(^+\) (DRK)}
\begin{itemize}
\item Caratteristiche:
  \begin{itemize}
  \item Apertura ritardata
  \item Nessun cancello di inattivazione
  \item Contribuisce alla ripolarizzazione
  \end{itemize}

\begin{figure}[h]
\centering
\includegraphics[width=0.5\textwidth]{canale_k.png}
\caption{Canale K\(^+\) delayed rectifier}
\label{fig:canale_k}
\end{figure}
\end{itemize}

% Sezione 4: Periodo refrattario
\section{Periodo refrattario}
\begin{itemize}
\item \textbf{Refrattarietà assoluta}:
  \begin{itemize}
  \item Canali Na\(^+\) inattivati
  \item Impossibilità di generare nuovi PA
  \end{itemize}

\item \textbf{Refrattarietà relativa}:
  \begin{itemize}
  \item Ripristino parziale dei canali Na\(^+\)
  \item Soglia di attivazione elevata
  \end{itemize}

\begin{figure}[h]
\centering
\includegraphics[width=0.6\textwidth]{refrattarieta.png}
\caption{Fasi refrattarie durante il PA}
\label{fig:refrattario}
\end{figure}
\end{itemize}

% Sezione 5: Modello di Hodgkin-Huxley
\section{Modello di Hodgkin-Huxley}
\begin{itemize}
\item Equazioni fondamentali:
\[
g_{Na} = \bar{g}_{Na} \cdot m^3 h \quad;\quad g_K = \bar{g}_K \cdot n^4
\]
\item Variabili:
  \begin{itemize}
  \item \( m \): Attivazione Na\(^+\)
  \item \( h \): Inattivazione Na\(^+\)
  \item \( n \): Attivazione K\(^+\)
  \end{itemize}

\begin{figure}[h]
\centering
\includegraphics[width=0.7\textwidth]{hodgkin_huxley.png}
\caption{Modello matematico della conduzione assonica}
\label{fig:hodgkin}
\end{figure}
\end{itemize}

\end{document}