\documentclass{article}
\usepackage[italian]{babel}
\usepackage{graphicx}
\usepackage{amsmath}
\usepackage{booktabs}
\usepackage{multirow}

\title{Generalità neuroni e glia}
\author{Orcam}
\date{}

\begin{document}

\maketitle

% Sezione 1: Struttura del neurone
\section{Il neurone}
\begin{itemize}
\item Componenti principali:
  \begin{itemize}
  \item \textbf{Dendriti}: Ricevono segnali
  \item \textbf{Soma}: Corpo cellulare con nucleo
  \item \textbf{Assone}: Trasmette impulsi
  \end{itemize}

\begin{figure}[h]
\centering
\includegraphics[width=1\textwidth]{Neuroscienze 2024-2025/Modulo I/Immagini Modulo I/Screenshot 2025-06-21 at 17-20-01 5. Sistema nervoso_generalità neuroni e glia.pdf.png}
\caption{Struttura di diversi neuroni}
\label{fig:neurone}
\end{figure}
\end{itemize}

% Sezione 2: Circuiti neuronali
\section{Circuiti neuronali}
\begin{itemize}
\item Organizzati in reti di comunicazione:
  \begin{itemize}
  \item Sinapsi tra terminali assonici e neuroni postsinaptici
  \item Trasmissione mediata da neurotrasmettitori
  \end{itemize}

\begin{figure}[h]
\centering
\includegraphics[width=1\textwidth]{Neuroscienze 2024-2025/Modulo I/Immagini Modulo I/Screenshot 2025-06-21 at 17-20-56 5. Sistema nervoso_generalità neuroni e glia.pdf.png}
\caption{Schema di un circuito neuronale}
\label{fig:circuito}
\end{figure}
\end{itemize}

% Sezione 3: Modello neurone artificiale
\section{Modello di McCulloch-Pitts (1943)}
\[
\text{Output} = f\left(\sum_{i=1}^n w_i x_i\right)
\]
\begin{itemize}
\item Elementi chiave:
  \begin{itemize}
  \item Input pesati (\(w_i\))
  \item Funzione di attivazione (\(f\))
  \item Soglia di attivazione
  \end{itemize}

\begin{figure}[h]
\centering
\includegraphics[width=1\textwidth]{Neuroscienze 2024-2025/Modulo I/Immagini Modulo I/Screenshot 2025-06-21 at 17-21-50 5. Sistema nervoso_generalità neuroni e glia.pdf.png}
\caption{Schema del neurone artificiale}
\label{fig:modello}
\end{figure}
\end{itemize}

% Sezione 4: Suddivisione SN
\section{Sistema Nervoso}
\begin{table}[h]
\centering
\begin{tabular}{ll}
\toprule
\textbf{Centrale (SNC)} & \textbf{Periferico (SNP)} \\
\midrule
Encefalo e midollo spinale & Nervi cranici e spinali \\
Elaborazione informazioni & Collegamento SNC-organi \\
\bottomrule
\end{tabular}
\caption{Confronto SNC e SNP}
\end{table}

% Sezione 5: Aree corticali
\section{Aree funzionali della corteccia}
\begin{itemize}
\item Principali divisioni:
  \begin{itemize}
  \item Corteccia motoria primaria (movimenti volontari)
  \item Area di Broca (produzione linguaggio)
  \item Corteccia somatosensoriale (percezione tattile)
  \item Corteccia visiva/uditiva
  \end{itemize}

\begin{figure}[h]
\centering
\includegraphics[width=1\textwidth]{Neuroscienze 2024-2025/Modulo I/Immagini Modulo I/Screenshot 2025-06-21 at 17-22-36 5. Sistema nervoso_generalità neuroni e glia.pdf.png}
\caption{Homunculus motorio e sensitivo}
\label{fig:homunculus}
\end{figure}
\end{itemize}

% Sezione 6: Sistema nervoso autonomo
\section{Sistema nervoso autonomo}
\begin{table}[h]
\centering
\begin{tabular}{lll}
\toprule
& Simpatico & Parasimpatico \\
\midrule
Funzione & "Lotta o fuga" & "Riposo e digestione" \\
Gangli & Paravertebrali & Vicino organi bersaglio \\
Neurotrasm. & Noradrenalina & Acetilcolina \\
\bottomrule
\end{tabular}
\caption{Confronto sistemi autonomi}
\end{table}

% Sezione 7: Cellule gliali
\section{Cellule gliali}
\begin{itemize}
\item Funzioni principali:
  \begin{itemize}
  \item \textbf{Oligodendrociti/Schwann}: Mielinizzazione
  \item \textbf{Astrociti}: Supporto metabolico
  \item \textbf{Microglia}: Funzione immunitaria
  \item \textbf{Ependimali}: Barriera emato-encefalica
  \end{itemize}

\begin{figure}[h]
\centering
\includegraphics[width=1\textwidth]{Neuroscienze 2024-2025/Modulo I/Immagini Modulo I/Screenshot 2025-06-21 at 17-25-09 5. Sistema nervoso_generalità neuroni e glia.pdf.png}
\caption{Cellule gliali nel SNC e SNP}
\label{fig:glia}
\end{figure}
\end{itemize}

% Sezione 8: Classificazione neuroni
\section{Tipi di neuroni}
\begin{itemize}
\item Per morfologia:
  \begin{itemize}
  \item Unipolari, bipolari, multipolari
  \end{itemize}
\item Per funzione:
  \begin{itemize}
  \item Sensoriali, motori, interneuroni
  \end{itemize}

\begin{figure}[h]
\centering
\includegraphics[width=1\textwidth]{Neuroscienze 2024-2025/Modulo I/Immagini Modulo I/Screenshot 2025-06-21 at 17-25-26 5. Sistema nervoso_generalità neuroni e glia.pdf.png}
\caption{Cellule gliali periferiche}
\label{fig:tipi}
\end{figure}
\end{itemize}

\end{document}
