\documentclass{article}
\usepackage[italian]{babel}
\usepackage{graphicx}
\usepackage{amsmath}
\usepackage{amssymb}
\usepackage{geometry}
\geometry{a4paper, margin=2.5cm}

\title{Propagazione del Potenziale d'Azione}
\author{Orcam}
\date{}

\begin{document}
\maketitle

\section{Propagazione dei Segnali Elettrici}
\subsection{Teoria del Cavo}
La propagazione passiva dei segnali elettrici è regolata dalle proprietà capacitive (\(C_m\)) e resistive (\(R_m\)) della membrana neuronale. Il modello include:
\begin{itemize}
    \item Componenti resistive (\(R_{m}\), \(R_p\)) e capacitive (\(C_m\)) in parallelo.
    \item Resistenza longitudinale (\(R_l\)) nel citoplasma.
\end{itemize}

% Inserire immagine: Modello circuito equivalente della membrana (Pag.1)

\subsection{Potenziali Elettrotonici}
\begin{itemize}
    \item Decadono con la distanza secondo: 
    \[
    V_x = V_0 e^{-x/\lambda} \quad \text{dove} \quad \lambda = \sqrt{\frac{R_m}{R_l}}
    \]
    \item \(\lambda\) (costante di spazio): distanza a cui \(V_m\) decade al 37\% di \(V_0\).
    \item Corrente dispersa attraverso \(R_m\) (membrana) e \(R_l\) (assoplasma).
\end{itemize}

% Inserire immagine: Decadimento esponenziale del potenziale (Pag.2)

\section{Potenziale d'Azione (PA)}
\subsection{Caratteristiche Principali}
\begin{itemize}
    \item \textbf{Non decade} con la distanza (meccanismo attivo).
    \item Generato da canali \(Na^+\) voltaggio-dipendenti che si aprono localmente.
    \item Ciclo rigenerativo (Hodgkin):
    \begin{enumerate}
        \item Depolarizzazione iniziale \(\rightarrow\) apertura canali \(Na^+\).
        \item Ingresso \(Na^+\) \(\rightarrow\) ulteriore depolarizzazione.
        \item Apertura canali \(K^+\) \(\rightarrow\) ripolarizzazione.
    \end{enumerate}
\end{itemize}

% Inserire immagine: Ciclo di Hodgkin (Pag.5)

\subsection{Propagazione Unidirezionale}
\begin{itemize}
    \item La corrente depolarizzante fluisce verso segmenti adiacenti.
    \item Periodo refrattario impedisce retropropagazione.
    \item Meccanismo a "domino": ogni segmento attiva il successivo.
\end{itemize}

% Inserire immagine: Propagazione unidirezionale (Pag.6-7)

\section{Fattori che Influenzano la Velocità di Conduzione}
\subsection{Diametro Assonale}
\begin{itemize}
    \item Assoni più grandi \(\rightarrow\) minore \(R_l\) \(\rightarrow\) maggiore \(\lambda\):
    \[
    \lambda \propto \sqrt{r} \quad (r = \text{raggio assonale})
    \]
    \item Velocità aumenta con il diametro (es.: assoni giganti di calamaro).
\end{itemize}

% Inserire immagine: Relazione diametro-velocità (Pag.14-15)

\subsection{Guaina Mielinica}
\begin{itemize}
    \item \textbf{Effetti della mielina}:
    \begin{itemize}
        \item Aumenta \(R_m\) e riduce \(C_m\) \(\rightarrow\) migliora efficienza.
        \item Conduzione saltatoria: PA si rigenerano solo ai nodi di Ranvier.
    \end{itemize}
    \item Velocità raggiunge 120 m/s (vs. 0.5-2 m/s in assoni amielinici).
\end{itemize}

% Inserire immagine: Conduzione saltatoria (Pag.19-20)

\section{Codifica dell'Intensità dello Stimolo}
\begin{itemize}
    \item \textbf{Legge del tutto-o-nulla}: l'ampiezza del PA è costante.
    \item L'intensità è codificata nella \textbf{frequenza} dei PA:
    \begin{itemize}
        \item Stimoli più intensi \(\rightarrow\) maggiore frequenza.
    \end{itemize}
\end{itemize}

% Inserire immagine: Frequenza PA e intensità stimolo (Pag.8-10)

% Inserire immagine: Confronto assoni mielinizzati/amielinici (Pag.21)

\end{document}