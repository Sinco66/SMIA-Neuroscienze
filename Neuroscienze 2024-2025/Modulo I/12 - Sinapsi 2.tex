\documentclass{article}
\usepackage[italian]{babel}
\usepackage{graphicx}
\usepackage{booktabs}
\usepackage{amsmath}
\usepackage{multirow}
\usepackage[margin=1in]{geometry}

\title{Riassunto: Sinapsi II}
\author{Orcam}
\date{}

\begin{document}

\maketitle

\section{Risposte Postsinaptiche}
\subsection{Tipi di Potenziali}
\begin{itemize}
\item \textbf{EPSP} (Potenziali Postsinaptici Eccitatori): depolarizzanti
\item \textbf{IPSP} (Potenziali Postsinaptici Inibitori): iperpolarizzanti
\end{itemize}

\subsection{Meccanismi di Generazione}
\begin{itemize}
\item Legame del neurotrasmettitore a recettori postsinaptici
\item Generazione di correnti ioniche sinaptiche
\item Natura eccitatoria/inibitoria determinata da:
  \begin{itemize}
  \item Tipo di recettore
  \item Gradienti ionici
  \item Non dal neurotrasmettitore in sé
  \end{itemize}
\end{itemize}

% Spazio per Figura 1: EPSP vs IPSP
\begin{figure}[h]
\centering
\includegraphics[width=0.7\textwidth]{figura1_epsp_ipsp}
\caption{Correnti ioniche nei potenziali postsinaptici}
\label{fig:epsp_ipsp}
\end{figure}

\section{Recettori Postsinaptici}
\subsection{Confronto Recettori}
\begin{table}[h]
\centering
\caption{Confronto tra recettori ionotropi e metabotropi}
\begin{tabular}{p{5cm}p{5cm}}
\toprule
\textbf{Recettori Ionotropi} & \textbf{Recettori Metabotropi} \\
\midrule
• Canali ionici ligando-dipendenti & • Attivano proteine G \\
• Risposta rapida (ms) & • Risposta lenta (s) \\
• Trasmissione diretta & • Trasduzione del segnale \\
• Esempi: nAChR, AMPA, NMDA & • Esempi: mAChR, recettori glutammato metabotropi \\
\bottomrule
\end{tabular}
\label{tab:recettori}
\end{table}

\subsection{Eterogeneità Recettoriale}
\begin{itemize}
\item Ogni neurotrasmettitore ha molteplici recettori
\item Esempi:
  \begin{itemize}
  \item \textbf{Acetilcolina}: Recettori nicotinici (ionotropi) e muscarinici (metabotropi)
  \item \textbf{Glutammato}: Recettori AMPA, NMDA (ionotropi), mGluR (metabotropi)
  \item \textbf{GABA}: Recettori GABA-A (ionotropi), GABA-B (metabotropi)
  \end{itemize}
\end{itemize}

% Spazio per Figura 2: Diversità recettoriale
\begin{figure}[h]
\centering
\includegraphics[width=0.8\textwidth]{figura2_diversita_recettori}
\caption{Eterogeneità dei recettori per acetilcolina e glutammato}
\label{fig:diversita_recettori}
\end{figure}

\section{Modulazione Sinaptica}
\subsection{Regolazione Recettori}
\begin{itemize}
\item Numero di recettori regolato dall'attività sinaptica
\item \textbf{Aumento attività}: Reclutamento recettori (esocitosi)
\item \textbf{Riduzione attività}: Rimozione recettori (endocitosi)
\end{itemize}

\subsection{Modulazione Presinaptica}
\begin{itemize}
\item \textbf{Sinapsi asso-assoniche}: Modulano il rilascio di neurotrasmettitore
\item \textbf{Meccanismi}:
  \begin{itemize}
  \item Facilitazione (aumento Ca\textsuperscript{2+})
  \item Inibizione (blocco Ca\textsuperscript{2+})
  \end{itemize}
\item Permette modulazione \textbf{selettiva} di bersagli specifici
\end{itemize}

% Spazio per Figura 3: Modulazione presinaptica
\begin{figure}[h]
\centering
\includegraphics[width=0.7\textwidth]{figura3_modulazione}
\caption{Meccanismo di inibizione presinaptica selettiva}
\label{fig:modulazione}
\end{figure}

\section{Integrazione Sinaptica}
\subsection{Sommazione Spaziale e Temporale}
\begin{table}[h]
\centering
\caption{Confronto sommazione spaziale e temporale}
\begin{tabular}{p{6cm}p{6cm}}
\toprule
\textbf{Sommazione Spaziale} & \textbf{Sommazione Temporale} \\
\midrule
• Input simultanei da sinapsi diverse & • Input successivi dalla stessa sinapsi \\
• Somma di correnti sinaptiche & • Somma di potenziali postsinaptici \\
• Efficacia dipende dalla costante di spazio λ & • Efficacia dipende dalla costante di tempo τ \\
\bottomrule
\end{tabular}
\label{tab:sommazione}
\end{table}

\subsection{Integrazione Neuronale}
\begin{itemize}
\item I potenziali postsinaptici sono graduati e locali
\item Devono propagarsi al cono di emergenza assonale per generare PA
\item Integrazione = somma pesata di input eccitatori e inibitori
\item Sinapsi inibitorie spesso localizzate sul soma per massima efficacia
\end{itemize}

% Spazio per Figura 4: Integrazione sinaptica
\begin{figure}[h]
\centering
\includegraphics[width=0.8\textwidth]{figura4_integrazione}
\caption{Processo di integrazione sinaptica nel neurone}
\label{fig:integrazione}
\end{figure}

\section{Fattori che Influenzano la Forza Sinaptica}
\begin{table}[h]
\centering
\caption{Fattori che determinano l'efficacia sinaptica}
\begin{tabular}{p{4cm}p{10cm}}
\toprule
\textbf{Fattore} & \textbf{Descrizione} \\
\midrule
\textbf{Presinaptici} & 
\begin{itemize}
\item Disponibilità neurotrasmettitore
\item Concentrazione Ca\textsuperscript{2+}
\item Attivazione recettori presinaptici (autorecettori)
\end{itemize} \\
\midrule
\textbf{Postsinaptici} & 
\begin{itemize}
\item Stato elettrico della membrana
\item Regolazione recettori (up/down-regulation)
\item Desensibilizzazione recettori
\end{itemize} \\
\midrule
\textbf{Generali} & 
\begin{itemize}
\item Area di contatto sinaptico
\item Degradazione enzimatica
\item Geometria della diffusione
\item Ricaptazione neurotrasmettitore
\end{itemize} \\
\bottomrule
\end{tabular}
\label{tab:forza_sinaptica}
\end{table}

\section{Neuromodulazione}
\subsection{Meccanismi}
\begin{itemize}
\item Diversi neurotrasmettitori rilasciati con tempi diversi:
  \begin{itemize}
  \item Neurotrasmettitori classici: effetto rapido (ms)
  \item Neuropeptidi: effetto lento (s)
  \end{itemize}
\item Esempio: Risposte colinergiche con diverse componenti temporali
\item Coinvolge sia recettori postsinaptici che presinaptici
\end{itemize}

% Spazio per Figura 5: Neuromodulazione
\begin{figure}[h]
\centering
\includegraphics[width=0.7\textwidth]{figura5_neuromodulazione}
\caption{Esempio di neuromodulazione in sinapsi colinergica}
\label{fig:neuromodulazione}
\end{figure}

\end{document}